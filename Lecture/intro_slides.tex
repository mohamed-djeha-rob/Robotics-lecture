\begin{frame}{Introduction}
	\begin{itemize}
		\item Robotics has a profound historical and cultural roots
		\item Constant attempts to make machines able to substitue humans in repeititve and tedious tasks
		\item Origin of \emph{‘Robot’} term
		\item Historical examples of robots: water clock (Harun Arrachid), Al Jazari, Turk chess player, Japanese robots, etc.
		\item Isaac Asimov sci-fi roman: robots as mechanical artifacts - Definition of \emph{‘Robotics’} (Science devoted to the study of robots) - Asimov's three fundamental laws:
		\begin{enumerate}
			\item A robot may not injure human beings 
			\item A robot must obey the orders given by human beings, except when such orders would conflict with the first law
			\item A robot must protect its own existence, as long as this protection does not conflict with the first and second laws
		\end{enumerate}
	\end{itemize}
\end{frame}
\begin{frame}
	\begin{itemize}
		\item A robot can be seen as a machine that is able to modify and interact with its environment in which it operates while respecting the Asimov's rules
		\item Components of a robot: 
		\begin{enumerate}
			\item Mechanical system
			\begin{itemize}
				\item Locomotion apparatus: wheels, crawler, legs, etc. 
				\item Manipulation apparatus: manipulator arms, end-effector, hands, section cap, etc.
			\end{itemize}
			\item Actuator: the organ by which the robot exert an action (either locomotion or manipulation)
			\item Sensors: the ability for the perception of the internal status (prioprioceptive sensors, e.g., joint-position sensors) or the external status (exteroceptive sensors, e.g., camera, force sensor)
			\item Control system: compute commands for the actuators based on the sensors information   
		\end{enumerate}		 
	\end{itemize}
\end{frame}
\begin{frame}{Robotic manipulator - Structure}
	\begin{itemize}
		\item \textbf{Robotic manipulator:} sequence of bodies (links) connected through articulation (joints).
		\item The rigid body tree (kinematic chain) is constituted by the \emph{arm}, the \emph{wrist} and the end-effector (hand)
		\item Two types of kinematic chain
		\begin{itemize}
			\item \emph{Open kinematic chain:} when there is only one kinematic chain between two ends. It can be a simple chain (robotic arm) or a tree kinematic chain (dual-arm robot, humanoid, quadruped, etc.) 
			\item \emph{Closed kinematic chain:} when a sequence of links forms a loop
		\end{itemize}
	%	\item Joints: can be either \emph{prismatic} or \emph{revolute}. A prismatic joint allows a relative translation between two bodies, whereas a revolute joint porduces a relative rotation
	%TODO add figures of the three types
	\end{itemize}
\end{frame}
\begin{frame}{Robotic manipulator - Joints}
		\begin{itemize}
		\item Basic joints are \emph{prismatic} or \emph{revolute}. 
		\item Prismatic joint allows a relative translation between two bodies 
		\item Revolute joint produces a relative rotation between two bodies
	\end{itemize}
	\begin{figure}
		\centering
		\includegraphics[width=0.6\columnwidth]{joints_type}
		\caption{Conventional representation of revloute and prismatic joints}
	\end{figure}
\end{frame}
\begin{frame}{Robotic manipulator - Joints}
	\begin{itemize}
		\item Any other types of joints can be modeled through prismatic or revolute joints, e.g., 
		\begin{itemize}
			\item Spherical joint: 3 revolute joints with concurrent axes (agile eye)
			\item Helical joint: combination of a translation $x$ and a rotation $\theta$ along the same axis: $x=p\theta$
		\end{itemize}
		\item Every allowed motion by a joint is a Degree of Freedom (DoF)
		\begin{itemize}
			\item Revolute, prismatic and helical joints: 1 DoF
			\item Spherical joint: 3 DoF
		\end{itemize}
		\item Each joint $\vectorQ$ has mechanical limits ($\vectorQ_{\min}$, $\vectorQ_{\max}$), velocity limits ($\vectorQdot_{\min}$, $\vectorQdot_{\max}$) and acceleration limits  ($\vectorQddot_{\min}$, $\vectorQddot_{\max}$)
	\end{itemize}
\end{frame}
\begin{frame}{Robotic manipulator - Workspace}
	\begin{itemize}
	%	\item Degree of Freedom (DoF) = number of possible motion - constraints imposed by the joint $\Rightarrow$ Each revolute or prismatic joint allows one DoF 
		\item Workspace: the space reachable by the robot's end-effector. Its shape and volume are defined by the robot structure and the mechanical joint limits.
		\begin{figure}
			\centering
			\includegraphics[width = 0.3\columnwidth]{Cartesian robot.png}
			\includegraphics[width = 0.3\columnwidth]{Cylindrical robot.png}
			\includegraphics[width = 0.3\columnwidth]{Spherical robot.png}
			\caption{Types of robots according to the shape of the workspace: \textbf{Cartesian} (left), \textbf{cylindrical} (middle) and \textbf{spherical} (right)}
		\end{figure}
	\end{itemize}
\end{frame}
\begin{frame}{Robotic manipulator - Repeatability and Payload}
	\begin{itemize}
		\item Repeatability: a measure of the robot ability to return to a previously reached position. It is measured in $mm$
		\item Payload: the maximum load the robot can widthstand statically.
	\end{itemize} 
\end{frame}

\begin{frame}{Robots are -\emph{almost}- everywhere !}	
	\begin{picture}(0,0)
		\put(0,70){Robots are becoming sophisticated, redundant and versatile}
		\put(10,-30){\begin{minipage}{0.3\columnwidth}
				\centering
				% \href{run:Figures/Quadruped_Redundancy.mp4}{
				% \includegraphics[height=\columnwidth]{Screenshot from 2025-06-10 16-43-33.png}} \\
				\embedvideo*{\includegraphics[width=1.5\columnwidth,height=3.8cm]{Screenshot from 2025-06-10 16-43-33.png}}{Figures/Quadruped_Redundancy.mp4}[autoplay=false,showGUI=true]\\
				\makebox{\hspace{1cm}\centering\small Armed \emph{Spot} quadruped robot}
		\end{minipage}}
		
		
		\put(230,-30){\begin{minipage}{0.3\columnwidth}
				\centering
				\embedvideo*{\includegraphics[width=1.5\columnwidth,height=3.8cm]{Screenshot from 2025-06-10 17-33-28.png}}{Figures/wheel-legged-robots.mp4}[autoplay=false,showGUI=true]\\
				% \href{run:Figures/wheel-legged-robots}{
				% \includegraphics[height=\columnwidth]{Screenshot from 2025-06-10 17-33-28.png}}
				% \\
				\makebox{\hspace{0cm}\centering\small Wheeled \emph{Anymal C} quadruped robot}
			%	\small{Wheeled quadruped robot}
		\end{minipage}}%{\includegraphics[height=0.25\columnwidth]{Screenshot from 2022-08-15 15-48-07.png}}
	\end{picture}
\end{frame}
\begin{frame}{Robots are -\emph{almost}- everywhere !}
	Robots are embedding multiple sensors	
	% \begin{picture}(0,0)
		% \put(0,70){}
		% \put(150,-20){
		% \begin{minipage}{\columnwidth}
			\begin{figure}
									\centering
								\embedvideo*{\includegraphics[width=0.59\columnwidth,height=5cm]{Screenshot from 2025-06-10 18-19-32.png}}{Figures/OceanOne.m4v}[autoplay=false,showGUI=true]\\
				\caption*{\emph{OceanOne$^\text{\emph{K}}$}  underwater humanoid robot}
			\end{figure}
				% \href{run:Figures/OceanOne.m4v}{
				% 	\includegraphics[height=\columnwidth]{Screenshot from 2025-06-10 18-19-32.png}}
				% \\

				% \makebox{\hspace{-1.25cm}\centering\small \emph{OceanOne$^\text{\emph{K}}$}  underwater humanoid robot}
			% \end{minipage}
			% }
				
		
%		\put(250,-30){\begin{minipage}{0.25\columnwidth}
%				\centering
%				\includegraphics[height=\columnwidth]{The-Eelume-AIAUV-courtesy-of-Eelume.png}\\
%				\small{Underwater exploration 
%				vehicle \\
%				(Basso et al., 2020)}
%			\end{minipage}}%{\includegraphics[height=0.25\columnwidth]{Screenshot from 2022-08-15 15-48-07.png}}
	% \end{picture}
\end{frame}
\begin{frame}{Robots are -\emph{almost}- everywhere !}	
	\begin{picture}(0,0)
		\put(0,70){Robots are able to accomplish complex missions...}
		\put(90,-20){\begin{minipage}{0.5\columnwidth}
				\centering
				% \href{run:Figures/Atlas Gets a Grip _ Boston Dynamics.mp4}{
					% \includegraphics[height=\columnwidth]{Screenshot from 2025-06-11 12-55-26.png}} 
						\embedvideo*{\includegraphics[width=1.20\textwidth,height=5cm]{Screenshot from 2025-06-11 12-55-26.png}}{Figures/Atlas Gets a Grip _ Boston Dynamics.mp4}[autoplay=false,showGUI=true]\\
				\makebox{\hspace{1.5cm}\centering\small \emph{Atlas} humanoid robot assisting workers}
		\end{minipage}}
	\end{picture}
\end{frame}
\begin{frame}{Robots are -\emph{almost}- everywhere !}	
	\begin{picture}(0,0)
		\put(0,70){... can cooperate ...}
		\put(90,-20){\begin{minipage}{0.5\columnwidth}
				\centering
				% \href{run:Figures/CooperativeLocomotion.m4v}{
				% 	\includegraphics[height=\columnwidth]{Screenshot from 2025-06-11 13-52-35.png}} \\
									\embedvideo*{\includegraphics[width=1.20\textwidth,height=5cm]{Screenshot from 2025-06-11 13-52-35.png}}{Figures/CooperativeLocomotion.m4v}[autoplay=false,showGUI=true]\\
				\makebox{\hspace{-0.8cm}\centering\small Two \emph{Go 1} quadruped robots cooperating to lift heavy objects}
		\end{minipage}}
	\end{picture}
\end{frame}
\begin{frame}[t]\frametitle{Robots are -\emph{almost}- everywhere !}
	\begin{picture}(0,0)
		\put(0,10){... and even replace humans in tedious tasks}
		\put(70,-90){\begin{minipage}{0.5\columnwidth}
				\centering
				% \href{run:Figures/CooperativeLocomotion.m4v}{
				% 	\includegraphics[height=\columnwidth]{Screenshot from 2025-06-11 13-52-35.png}} \\
			\embedvideo*{\includegraphics[width=1.3\textwidth,height=5.5cm]{Screenshot from 2025-06-11 13-52-35.png}}{Figures/Motivation-IAM.mp4}[autoplay=false,showGUI=true]\\
				\makebox{\hspace{1.8cm}\centering\small Results from the European Project I.AM}
		\end{minipage}}
	\end{picture}

\end{frame}
\begin{frame}{Robots are -\emph{almost}- everywhere !}	
	\begin{picture}(0,0)
		\put(0,70){... but have to satisfy some hard constraints}
		\put(00,-20){\begin{minipage}{0.5\columnwidth}
				\centering
				% \href{run:Figures/Control Barrier Function based safe navigation -- experiment.mp4}{
				% 	\includegraphics[height=\columnwidth]{Screenshot from 2025-06-11 16-33-50.png}} \\
										\embedvideo*{\includegraphics[width=\columnwidth,height=4.3cm]{Screenshot from 2025-06-11 16-33-50.png}}{Figures/Control Barrier Function based safe navigation -- experiment.mp4}[autoplay=false,showGUI=true]\\

				\makebox{\hspace{-0.2cm}\centering\small Getting to the target while avoiding collision}
		\end{minipage}}
		\put(220,-20){\begin{minipage}{0.5\columnwidth}
				\centering
				% \href{run:Figures/Safe Behavior through Control Barrier Functions.mp4}{
				% 	\includegraphics[height=\columnwidth]{Screenshot from 2025-06-11 16-33-28.png}} \\
					\embedvideo*{\includegraphics[width=\columnwidth,height=4.3cm]{Screenshot from 2025-06-11 16-33-28.png}}{Figures/Safe Behavior through Control Barrier Functions.mp4}[autoplay=false,showGUI=true]\\

				\makebox{\hspace{-0.5cm}\centering\small Keeping upright position}
		\end{minipage}}
	\end{picture}
\end{frame}
%\begin{frame}{Motivation - One control framework, many applications}
%	\vspace{-0.5cm}
%%	\begin{multicols}{2}
%		% Left column
%		\begin{tcolorbox}[colback=gray!5!white, colframe=gray!75!black, title=One robotic application]
%			\begin{itemize}\small
%				\item Use high DoF robots
%				\item Achieve complex maneuvers
%				\item Rely on sensors
%				\item Satisfy constraints: safety specifications, hardware limits, etc.
%			\end{itemize}
%		\end{tcolorbox}
%		%  \columnbreak $\overset{\rm Objective}{\longrightarrow}$
%		
%		% Right column
%	%	\columnbreak
%		\vspace{-0.5cm}
%		\begin{tcolorbox}[colback=green!5!white, colframe=green!70!black, title=Objective]
%			%	\textbf{Control framework}
%		\small	Is it possible to have \textbf{‘one control framework’} that can be used for \textbf{‘many robotic applications’}?
%		\end{tcolorbox}		
%%	\end{multicols}
%\end{frame}
%\begin{frame}{Motivation - One control framework, many applications}
%	The control framework should be: 
%%	\begin{multicols}{1}
%		% Left column		
%		\begin{tcolorbox}[colback=green!5!white, colframe=green!70!black]%-è, title=Need for a control framework]
%		%	\textbf{Control framework}
%			\begin{itemize} 
%				\item \textbf{Consistent} enough to render the required application complexity
%				\item \textbf{General} enough to encompass new applications
%				\item \textbf{Simple} enough for the user
%			\end{itemize}
%		\end{tcolorbox}		
%%	\end{multicols}
%\end{frame}
%
%\begin{frame}{Motivation - Control framework construction}
%	\begin{figure}%[t!]
%		\centering
%		\begin{tikzpicture}[auto, node distance=2cm,>=latex']
%			\node  [block](CRA) at (-1, 12) {Complex robotic application};	
%			\node (COP) at (2.2, 9) {Control objective primitives};
%			\node [block] (TC) at (2.2, 8)  {\begin{varwidth}{0.24\linewidth}
%					\begin{itemize}
%						\item Tasks
%						\item Constraints
%					\end{itemize}		
%			\end{varwidth}};
%			\node (hardware) at (7.4, 9) {Hardware};
%			\node [block] (RS) at (7.4, 8) {\begin{varwidth}{0.2\linewidth}
%					\begin{itemize}
%						\item Robots
%						\item Sensors
%					\end{itemize}	
%			\end{varwidth}};
%			\node (PT) at (-3.5, 9) {Parameters tuning};
%			\node [block] (PC) at (-3.5, 8) {\begin{varwidth}{0.31\linewidth}
%					\begin{itemize}
%						\item Priority
%						\item Convergence rate
%					\end{itemize}	
%			\end{varwidth}};
%			\node (w1) at (4.8, 7) {\small Atomic decomposition};
%			\node (w2) at (-0.6, 7) {\small Combination};
%			\node [tmp] at (-1.05, 12) (tmp) {}; 
%		
%			\draw [->] ([yshift=0pt]RS.west) -- node[below,pos=0.54]{}(TC);
%			\draw [->] ([yshift=0pt]TC.west) -- node[below,pos=0.54]{}(PC);
%		%	\draw [->] ([xshift=-60pt]PC.north) |- node[above,pos=0.7]{\small Complexity rendering}(tmp);
%			
%		\end{tikzpicture}
%	\end{figure}	
%\end{frame}
%\begin{frame}{Motivation - Control framework construction}
%	\begin{figure}
%		\centering
%		\begin{tikzpicture}[auto, node distance=2cm,>=latex']
%			% Nodes
%			\node [block] (CRA) at (-1, 12) {Complex robotic application};
%		%	\node [sum] (comp) at (4,11) {}; 
%			\node [block] (Sim) at (-1,10.5) {Simulation};
%			\node (COP) at (2.2, 9) {Control objective primitives};
%		%	\node [block] (adjust) at (6,11) {Adjustment};
%			\node [block] (TC) at (2.2, 8) {\begin{varwidth}{0.24\linewidth}
%					\begin{itemize}
%						\item Tasks
%						\item Constraints
%					\end{itemize}		
%			\end{varwidth}};
%			\node (hardware) at (7.4, 9) {Hardware};
%			\node [block] (RS) at (7.4, 8) {\begin{varwidth}{0.2\linewidth}
%					\begin{itemize}
%						\item Robots
%						\item Sensors
%					\end{itemize}	
%			\end{varwidth}};
%			\node (PT) at (-3.5, 9) {Parameters tuning};
%			\node [block] (PC) at (-3.5, 8) {\begin{varwidth}{0.31\linewidth}
%					\begin{itemize}
%						\item Priority
%						\item Convergence rate
%					\end{itemize}	
%			\end{varwidth}};
%			\node (w1) at (4.8, 7) {\small Atomic decomposition};
%			\node (w2) at (-0.6, 7) {\small Combination};
%			\node [tmp] at (-2.3, 10.5) (tmp) {}; 
%%			\node [tmp] at (7,10.5) (tmp2) {};
%		%	\node (w3) at (3, 10.2) {\small Complexity rendering};
%		%	\node (w4) at (4, 12.2) {\small Desired complexity};
%			% Existing arrows
%			\draw [->] (RS.west) -- (TC.east);
%			\draw [->] (TC.west) -- (PC.east);
%			\draw [->] ([xshift=-60pt]PC.north) |- (tmp);
%		%	\draw [->] (Sim.east) -| (comp.south);
%		%	\draw [->] (CRA.east) -| (comp.north);
%		%	\draw [->] (comp.east) -- (adjust.west);
%			% NEW: diagonal tuning arrows above blocks
%%			\foreach \n in {TC,RS,PC}{
%%				\draw[->,line width=2.5pt,red,opacity=0.55, dashed]  ([yshift=5pt]\n.north east) -- ([yshift=-5pt]\n.south west);
%%			}
%		\end{tikzpicture}
%	\end{figure}	
%\end{frame}
%\begin{frame}{Motivation - Control framework construction}
%	\begin{figure}
%		\centering
%		\begin{tikzpicture}[auto, node distance=2cm,>=latex']
%			% Nodes
%			\node [block] (CRA) at (-1, 12) {Complex robotic application};
%			\node [sum] (comp) at (4,11) {}; 
%			\node [block] (adjust) at (6,11) {\color{red}Adjustment};
%			\node [block] (Sim) at (-1,10.5) {Simulation};
%			\node (COP) at (2.2, 9) {Control objective primitives};
%			\node [block] (TC) at (2.2, 8) {\begin{varwidth}{0.24\linewidth}
%					\begin{itemize}
%						\item Tasks
%						\item Constraints
%					\end{itemize}		
%			\end{varwidth}};
%			\node (hardware) at (7.4, 9) {Hardware};
%			\node [block] (RS) at (7.4, 8) {\begin{varwidth}{0.2\linewidth}
%					\begin{itemize}
%						\item Robots
%						\item Sensors
%					\end{itemize}	
%			\end{varwidth}};
%			\node (PT) at (-3.5, 9) {Parameters tuning};
%			\node [block] (PC) at (-3.5, 8) {\begin{varwidth}{0.31\linewidth}
%					\begin{itemize}
%						\item Priority
%						\item Convergence rate
%					\end{itemize}	
%			\end{varwidth}};
%			\node (w1) at (4.8, 7) {\small Atomic decomposition};
%			\node (w2) at (-0.6, 7) {\small Combination};
%			\node [tmp] at (-2.3, 10.5) (tmp) {}; 
%			%			\node [tmp] at (7,10.5) (tmp2) {};
%			\node (w3) at (3, 10.2) {\small Rendered Complexity};
%			\node (w4) at (4, 12.2) {\small Desired complexity};
%			% Existing arrows
%			\draw [->] (RS.west) -- (TC.east);
%			\draw [->] (TC.west) -- (PC.east);
%			\draw [->] ([xshift=-60pt]PC.north) |- (tmp);
%			\draw [->] (Sim.east) -| (comp.south);
%			\draw [->] (CRA.east) -| (comp.north);
%			\draw [->] (comp.east) -- (adjust.west);
%			% NEW: diagonal tuning arrows above blocks
%			\foreach \n in {TC,RS,PC}{
%				\draw[->,line width=2.5pt,red,opacity=0.55, dashed]  ([yshift=5pt]\n.north east) -- ([yshift=-5pt]\n.south west);
%			}
%		\end{tikzpicture}
%	\end{figure}	
%\end{frame}
%\begin{frame}{Motivation - Control framework construction}
%	Example: Moving the base of a quadruped robot
%	\begin{minipage}{0.51\columnwidth}
%		\begin{figure}
%			\includegraphics[width=\columnwidth]{Screenshot from 2025-06-10 16-43-33}
%		\end{figure}
%	\end{minipage}
%	\hfill
%	\begin{minipage}{0.48\columnwidth}
%		 \invisible{
%		\begin{itemize}
%			\item \textbf{Task:} move the base in the manipulability space
%			\item \textbf{Constraints:} 
%			\begin{itemize}
%				\item Keep end-effector fixed
%				\item Keep non-slipping contacts
%				\item Avoid self-collision
%				\item Respect joint position/velocity/torque constraints
%				\item Keep CoM projection inside the static equilibrium region
%			\end{itemize}
%		\end{itemize}}
%	\end{minipage}
%\end{frame}
%\begin{frame}{Motivation - Control framework construction}
%	Example: Moving the base of a quadruped robot
%	\begin{minipage}{0.51\columnwidth}
%		\begin{figure}
%			\includegraphics[width=\columnwidth]{multi-task example.pdf}
%		\end{figure}
%	\end{minipage}
%	\hfill
%	\begin{minipage}{0.48\columnwidth}
%		\begin{itemize}
%			\item \textbf{Task:} move the base in the manipulability space
%			\item \textbf{Constraints:} 
%			\begin{itemize}
%				\item Keep end-effector fixed
%				\item Keep non-slipping contacts
%				\item Avoid self-collision
%				\item Respect joint position/velocity/torque constraints
%				\item Keep CoM projection inside the static equilibrium region
%			\end{itemize}
%		\end{itemize}
%		
%	\end{minipage}
%\end{frame}
%\begin{frame}{Motivation - Control framework construction}
%\textbf{	One \emph{elegant} solution: Quadratic Programming paradigm }
%	\begin{align*}
%		\left.\begin{matrix}
%			\textit{Multi}\text{-robot} \\
%			\textit{Multi}\text{-task} \\
%			\textit{Multi}\text{-sensor}
%		\end{matrix}
%		\right\}		
%		\text{Task-space Quadratic Programming Control}
%	\end{align*}
%\end{frame}
%\begin{frame}{QP Control - Historical overview}
%	%\begin{columns}
%		%\column{0.5\linewidth}
%		\begin{itemize}
%			\item  \textbf{1960s:} Most works focused on numerical algorithms for QP solving
%			\item \textbf{1970s:} Emergence of Model Predictive Control (MPC) - Use of QP as a \emph{tool} for solving the MPC problem~\cite{richalet1978automatica}
%			\item \textbf{1980s:} First use QP for trajectory planning with collision avoidance for redundant manipulator control as an alternative to artificial potential fields method~\cite{faverjon1987icra}
%			\item \textbf{1990s:} Premises of QP as a \emph{‘controller’} - Standard QP formulations are posed for redundant manipulator control with joints constraints~\cite{cheng1994tra,park1998icra}
%		\end{itemize}
%\end{frame}
%\begin{frame}{QP Control - Historical overview}
%		\begin{itemize}
%			
%			\item \textbf{(2000-2010)s:} Multi-objective control - Use of QP for motion generation in graphical animations of robots in multi-contact~\cite{zhang2004transactionsonSysManCyb2,abe2007siggraph,macchietto2009siggraph,salini2010springer} 
%			\item \textbf{(2010-Now):} Golden era of QP control - Real-time QP control of large classes of robots (manipulators, humanoids, quadrupeds, drones, unicycles, etc.) with real world application - New control strategy enabled - Breakthrough theoretical contributions 
%			\item \textbf{(2021-Now):} Real-time implementation of whole-body MPC 
%		\end{itemize}
%		%\column{0.4\linewidth}
%% 		\begin{tikzpicture}[spy using outlines={rectangle,thick,red,fill=red,opacity=0.2,
%  %                     lens={scale=2.5}, width=2.5cm, height=1cm, connect spies}]
%  %   % Place main figure
%  %   \node (image) {\includegraphics[width=0.5\columnwidth]{Figures/mpc_history_0.png}};
%  %   % Spy on coordinates (x,y) in the image coordinate system
%  %   % Adjust (x,y) to your region of interest
%  %   \spy on (0.12,0.7) in node [fill=white] at (4,1);
%  %   % \fill[red,opacity=0.2] (0.41,0.8) rectangle (0.59,0.70);
%  % \end{tikzpicture}
%% 	\end{columns}
%\end{frame}
\begin{frame}{Outline}
	\tableofcontents
\end{frame}
%\begin{frame}{Motivation - Control framework construction}
%One solution: \textbf{Quadratic Programming} paradigm 
%	\begin{align*}
%		\begin{matrix}
%			\textit{Multi-robots} \\
%			\textit{Multi-tasks} \\
%			\textit{Multi-sensors}
%		\end{matrix}
%		\left\}\begin{matrix}
%		\begin{split}
%			\underset{\decisionVar}{\min}&\frac{1}{2}\decisionVar^T\mathbf{H}\decisionVar +\bm{h}^T\decisionVar \\
%			\rm{S.t:~}&\mathbf{C}_{\rm ineq}\decisionVar\leq\bm{d}_{\rm ineq}\\
%			&\mathbf{C}_{\rm eq}\decisionVar=\bm{d}_{\rm eq}
%		\end{split}
%	\end{matrix}\right.
%	\end{align*}
%\end{frame}