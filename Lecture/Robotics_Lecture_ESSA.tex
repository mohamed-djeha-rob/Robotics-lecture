\PassOptionsToPackage{table,dvipsnames,svgnames}{xcolor}
\documentclass[14pt,aspectratio=169]{beamer}
%\documentclass[14pt,  aspectratio=169]{beamer}  

\usepackage[default,oldstyle,scale=0.95]{opensans}
\usepackage[utf8]{inputenc}
\usepackage{xparse}
\usepackage[bigfiles]{pdfbase}
\usepackage{xkeyval} % For handling key-value pairs

\makeatletter
\define@boolkey{embedvideo}{autoplay}{\def\embedvideoautoplay{#1}}
\define@boolkey{embedvideo}{showGUI}{\def\embedvideoshowGUI{#1}}
\presetkeys{embedvideo}{autoplay=true,showGUI=true}{}

\ExplSyntaxOn

\NewDocumentCommand\embedvideo{smmO{autoplay=true,showGUI=true}}{
  \group_begin:
  \leavevmode

  % Parse the options (autoplay, showGUI)
  \setkeys{embedvideo}{#4}

  \tl_if_exist:cTF{file_\file_mdfive_hash:n{#3}}{
    \tl_set_eq:Nc\video{file_\file_mdfive_hash:n{#3}}
  }{
    \IfFileExists{#3}{}{\GenericError{}{File~`#3'~not~found}{}{}}
    \pbs_pdfobj:nnn{}{fstream}{{}{#3}}
    \pbs_pdfobj:nnn{}{dict}{
      /Type/Filespec/F~(#3)/UF~(#3)
      /EF~<</F~\pbs_pdflastobj:>>
    }
    \tl_set:Nx\video{\pbs_pdflastobj:}
    \tl_gset_eq:cN{file_\file_mdfive_hash:n{#3}}\video}

  % Define the video parameters with the GUI and autoplay options
  \pbs_pdfobj:nnn{}{dict}{
    /Type/RichMediaInstance/Subtype/Video
    /Asset~\video/Params~<</FlashVars (source=#3&
      skinAutoHide=\embedvideoshowGUI&
      skin= skin1&
      controlBarMode=floating&
      controls=\embedvideoshowGUI&
      skinBackgroundColor=0x5F5F5F&
      skinBackgroundAlpha=0&
      speed=10&
      loop=true&
      volume=100&
      replay=true&
      stop=true&
      navigationSlider=true&
      autoplay=\embedvideoautoplay)>>
  }

  \pbs_pdfobj:nnn{}{dict}{
    /Type/RichMediaConfiguration/Subtype/Video
    /Instances~[\pbs_pdflastobj:]
  }

  \pbs_pdfobj:nnn{}{dict}{
    /Type/RichMediaContent
    /Assets~<<
      /Names~[(#3)~\video]
    >>
    /Configurations~[\pbs_pdflastobj:]
  }

  \tl_set:Nx\rmcontent{\pbs_pdflastobj:}

  \pbs_pdfobj:nnn{}{dict}{
    /Activation~<<
      /Condition/\IfBooleanTF{#1}{PV}{XA}
      /Presentation~<</Style/Embedded>>
    >>
    /Deactivation~<</Condition/PI>>
  }

  \hbox_set:Nn\l_tmpa_box{#2}
  \tl_set:Nx\l_box_wd_tl{\dim_use:N\box_wd:N\l_tmpa_box}
  \tl_set:Nx\l_box_ht_tl{\dim_use:N\box_ht:N\l_tmpa_box}
  \tl_set:Nx\l_box_dp_tl{\dim_use:N\box_dp:N\l_tmpa_box}
  \pbs_pdfxform:nnnnn{1}{1}{}{}{\l_tmpa_box}

  \pbs_pdfannot:nnnn{\l_box_wd_tl}{\l_box_ht_tl}{\l_box_dp_tl}{
    /Subtype/RichMedia
    /BS~<</W~0/S/S>>
    /Contents~(embedded~video~file:#3)
    /NM~(rma:#3)
    /AP~<</N~\pbs_pdflastxform:>>
    /RichMediaSettings~\pbs_pdflastobj:
    /RichMediaContent~\rmcontent}
  \phantom{#2}
  \group_end:
}

\ExplSyntaxOff
\makeatother

\usepackage{graphicx}
\usepackage{tcolorbox}
\usepackage{multicol}
\usepackage{amsmath,amsfonts}
\usepackage{tikz}
\usetikzlibrary{shapes,arrows,positioning,calc,fit,decorations.markings,spy}
\usepackage{tikz-cd}
\usepackage{pdfpages}
%\usepackage[usenames,table,dvipsnames,svgnames]{xcolor}
\graphicspath{{Figures/}}
\usepackage{media9}
\usepackage{hyperref}
\usepackage{subfiles}
\usepackage{varwidth}
\usepackage{bm}
\usepackage{caption}
\usepackage[stable]{footmisc}
\usepackage{cleveref}
\usepackage[percent]{overpic}
% \setbeamertemplate{footnote}{%
%   \hangpara{2em}{1}%
%   \makebox[2em][l]{\insertfootnotemark}\footnotesize\insertfootnotetext\par%
% }
%commands
\def\decisionVar{\boldsymbol{\chi}}
\def\matriceA{\mathbf{A}}
\def\matriceB{\mathbf{B}}
\def\matriceC{\mathbf{C}}
\def\matriceD{\mathbf{D}}
\def\matriceH{\mathbf{H}}
\def\matriceR{\mathbf{R}}
\def\matriceRdot{\mathbf{\dot{R}}}
\def\matriceQ{\mathbf{Q}}
\def\matriceK{\mathbf{K}}
\def\matriceM{\mathbf{M}}
\def\matriceS{\mathbf{S}}
\def\matriceJ{\mathbf{J}}
\def\matriceT{\mathbf{T}}
\def\matriceI{\mathbf{I}}
\def\vectorB{\bm{b}}
\def\vectorD{\bm{d}}
\def\vectorH{\bm{h}}
\def\vectorX{\bm{x}}
\def\vectorXMaj{\bm{X}}
\def\vectorU{\bm{u}}
\def\vectorS{\bm{s}}
\def\vectorY{\bm{y}}
\def\vectorZ{\bm{z}}
\def\vectorV{\bm{v}}
\def\vectorE{\bm{e}}
\def\vectorC{\bm{c}}
\def\vectorF{\bm{f}}
\def\vectorQ{\bm{q}}
\def\vectorO{\bm{o}}
\def\vectorP{\bm{p}}
\def\vectorZero{\bm{0}}
\def\vectorOmega{\boldsymbol{\omega}}
\def\vectorTau{\boldsymbol{\tau}}
\def\vectorXdot{\bm{\dot{x}}}
\def\vectorYdot{\bm{\dot{y}}}
\def\vectorZdot{\bm{\dot{z}}}
\def\vectorEdot{\bm{\dot{e}}}
\def\vectorQdot{\bm{\dot{q}}}
\def\vectorQddot{\bm{\ddot{q}}}
\def\inR{\in\mathbb{R}}
\def\setC{{\cal C}}
\def\setR{{\cal R}}
\def\setS{{\cal S}}
\newcommand{\norm}[1]{\begin{Vmatrix}
		#1
\end{Vmatrix}}
\usetheme{CambridgeUS}

\usefonttheme[onlymath]{serif}
\tikzset{
	block/.style = {draw, fill=white, rectangle, minimum height=1.5em, minimum width=2em},
	tmp/.style  = {coordinate}, 
	sum/.style= {draw, fill=white, circle, node distance=1cm},
	input/.style = {coordinate},
	output/.style= {coordinate},
	pinstyle/.style = {pin edge={to-,thin,black}},
	dotted_block/.style={draw=black!50!white, line width=1.5pt, dash pattern=on 3pt off 3pt on 3pt off 3pt, inner ysep=3mm,inner xsep=2mm, rectangle, rounded corners}
}

\title[Introduction à la robotique industrielle]{Robots industriels: \\ Opportunités et défis d'exploitation}
\author[Dr Alouane, Dr Djeha]{Dr Alouane Mohamed Amine, Dr Djeha Mohamed}
\institute[LCSCS]{UER SAI - LCSCS}
\date{ESSAT Seminar -- \today}
\AtBeginSection[]{
	\begin{frame}
		\centering
		\vfill
		\usebeamerfont{section title}\LARGE\insertsection
		\vfill
	\end{frame}
}
\AtBeginDocument{%
	\hypersetup{colorlinks=true, citecolor=SeaGreen}%
	\makeatletter
	\def\@citecolor{SeaGreen}%
	\makeatother
}
\begin{document}
	
% Title Slide
\begin{frame}
	\titlepage%~\cite{richalet1978automatica}
\end{frame}	
\section*{Introduction to industrial robotics}
\begin{frame}{Introduction}
	\begin{itemize}
		\item Robotics has a profound historical and cultural roots
		\item Constant attempts to make machines able to substitue humans in repeititve and tedious tasks
		\item Origin of \emph{‘Robot’} term
		\item Historical examples of robots: water clock (Harun Arrachid), Al Jazari, Turk chess player, Japanese robots, etc.
		\item Isaac Asimov sci-fi roman: robots as mechanical artifacts - Definition of \emph{‘Robotics’} (Science devoted to the study of robots) - Asimov's three fundamental laws:
		\begin{enumerate}
			\item A robot may not injure human beings 
			\item A robot must obey the orders given by human beings, except when such orders would conflict with the first law
			\item A robot must protect its own existence, as long as this protection does not conflict with the first and second laws
		\end{enumerate}
	\end{itemize}
\end{frame}
\begin{frame}
	\begin{itemize}
		\item A robot can be seen as a machine that is able to modify and interact with its environment in which it operates while respecting the Asimov's rules
		\item Components of a robot: 
		\begin{enumerate}
			\item Mechanical system
			\begin{itemize}
				\item Locomotion apparatus: wheels, crawler, legs, etc. 
				\item Manipulation apparatus: manipulator arms, end-effector, hands, section cap, etc.
			\end{itemize}
			\item Actuator: the organ by which the robot exert an action (either locomotion or manipulation)
			\item Sensors: the ability for the perception of the internal status (prioprioceptive sensors, e.g., joint-position sensors) or the external status (exteroceptive sensors, e.g., camera, force sensor)
			\item Control system: compute commands for the actuators based on the sensors information   
		\end{enumerate}		 
	\end{itemize}
\end{frame}
\begin{frame}{Robotic manipulator - Structure}
	\begin{itemize}
		\item \textbf{Robotic manipulator:} sequence of bodies (links) connected through articulation (joints).
		\item The rigid body tree (kinematic chain) is constituted by the \emph{arm}, the \emph{wrist} and the end-effector (hand)
		\item Two types of kinematic chain
		\begin{itemize}
			\item \emph{Open kinematic chain:} when there is only one kinematic chain between two ends. It can be a simple chain (robotic arm) or a tree kinematic chain (dual-arm robot, humanoid, quadruped, etc.) 
			\item \emph{Closed kinematic chain:} when a sequence of links forms a loop
		\end{itemize}
	%	\item Joints: can be either \emph{prismatic} or \emph{revolute}. A prismatic joint allows a relative translation between two bodies, whereas a revolute joint porduces a relative rotation
	%TODO add figures of the three types
	\end{itemize}
\end{frame}
\begin{frame}{Robotic manipulator - Joints}
		\begin{itemize}
		\item Basic joints are \emph{prismatic} or \emph{revolute}. 
		\item Prismatic joint allows a relative translation between two bodies 
		\item Revolute joint produces a relative rotation between two bodies
	\end{itemize}
	\begin{figure}
		\centering
		\includegraphics[width=0.6\columnwidth]{joints_type}
		\caption{Conventional representation of revloute and prismatic joints}
	\end{figure}
\end{frame}
\begin{frame}{Robotic manipulator - Joints}
	\begin{itemize}
		\item Any other types of joints can be modeled through prismatic or revolute joints, e.g., 
		\begin{itemize}
			\item Spherical joint: 3 revolute joints with concurrent axes (agile eye)
			\item Helical joint: combination of a translation $x$ and a rotation $\theta$ along the same axis: $x=p\theta$
		\end{itemize}
		\item Every allowed motion by a joint is a Degree of Freedom (DoF)
		\begin{itemize}
			\item Revolute, prismatic and helical joints: 1 DoF
			\item Spherical joint: 3 DoF
		\end{itemize}
		\item Each joint $\vectorQ$ has mechanical limits ($\vectorQ_{\min}$, $\vectorQ_{\max}$), velocity limits ($\vectorQdot_{\min}$, $\vectorQdot_{\max}$) and acceleration limits  ($\vectorQddot_{\min}$, $\vectorQddot_{\max}$)
	\end{itemize}
\end{frame}
\begin{frame}{Robotic manipulator - Workspace}
	\begin{itemize}
	%	\item Degree of Freedom (DoF) = number of possible motion - constraints imposed by the joint $\Rightarrow$ Each revolute or prismatic joint allows one DoF 
		\item Workspace: the space reachable by the robot's end-effector. Its shape and volume are defined by the robot structure and the mechanical joint limits.
		\begin{figure}
			\centering
			\includegraphics[width = 0.3\columnwidth]{Cartesian robot.png}
			\includegraphics[width = 0.3\columnwidth]{Cylindrical robot.png}
			\includegraphics[width = 0.3\columnwidth]{Spherical robot.png}
			\caption{Types of robots according to the shape of the workspace: \textbf{Cartesian} (left), \textbf{cylindrical} (middle) and \textbf{spherical} (right)}
		\end{figure}
	\end{itemize}
\end{frame}
\begin{frame}{Robotic manipulator - Repeatability and Payload}
	\begin{itemize}
		\item Repeatability: a measure of the robot ability to return to a previously reached position. It is measured in $mm$
		\item Payload: the maximum load the robot can widthstand statically.
	\end{itemize} 
\end{frame}

\begin{frame}{Robots are -\emph{almost}- everywhere !}	
	\begin{picture}(0,0)
		\put(0,70){Robots are becoming sophisticated, redundant and versatile}
		\put(10,-30){\begin{minipage}{0.3\columnwidth}
				\centering
				% \href{run:Figures/Quadruped_Redundancy.mp4}{
				% \includegraphics[height=\columnwidth]{Screenshot from 2025-06-10 16-43-33.png}} \\
				\embedvideo*{\includegraphics[width=1.5\columnwidth,height=3.8cm]{Screenshot from 2025-06-10 16-43-33.png}}{Figures/Quadruped_Redundancy.mp4}[autoplay=false,showGUI=true]\\
				\makebox{\hspace{1cm}\centering\small Armed \emph{Spot} quadruped robot}
		\end{minipage}}
		
		
		\put(230,-30){\begin{minipage}{0.3\columnwidth}
				\centering
				\embedvideo*{\includegraphics[width=1.5\columnwidth,height=3.8cm]{Screenshot from 2025-06-10 17-33-28.png}}{Figures/wheel-legged-robots.mp4}[autoplay=false,showGUI=true]\\
				% \href{run:Figures/wheel-legged-robots}{
				% \includegraphics[height=\columnwidth]{Screenshot from 2025-06-10 17-33-28.png}}
				% \\
				\makebox{\hspace{0cm}\centering\small Wheeled \emph{Anymal C} quadruped robot}
			%	\small{Wheeled quadruped robot}
		\end{minipage}}%{\includegraphics[height=0.25\columnwidth]{Screenshot from 2022-08-15 15-48-07.png}}
	\end{picture}
\end{frame}
\begin{frame}{Robots are -\emph{almost}- everywhere !}
	Robots are embedding multiple sensors	
	% \begin{picture}(0,0)
		% \put(0,70){}
		% \put(150,-20){
		% \begin{minipage}{\columnwidth}
			\begin{figure}
									\centering
								\embedvideo*{\includegraphics[width=0.59\columnwidth,height=5cm]{Screenshot from 2025-06-10 18-19-32.png}}{Figures/OceanOne.m4v}[autoplay=false,showGUI=true]\\
				\caption*{\emph{OceanOne$^\text{\emph{K}}$}  underwater humanoid robot}
			\end{figure}
				% \href{run:Figures/OceanOne.m4v}{
				% 	\includegraphics[height=\columnwidth]{Screenshot from 2025-06-10 18-19-32.png}}
				% \\

				% \makebox{\hspace{-1.25cm}\centering\small \emph{OceanOne$^\text{\emph{K}}$}  underwater humanoid robot}
			% \end{minipage}
			% }
				
		
%		\put(250,-30){\begin{minipage}{0.25\columnwidth}
%				\centering
%				\includegraphics[height=\columnwidth]{The-Eelume-AIAUV-courtesy-of-Eelume.png}\\
%				\small{Underwater exploration 
%				vehicle \\
%				(Basso et al., 2020)}
%			\end{minipage}}%{\includegraphics[height=0.25\columnwidth]{Screenshot from 2022-08-15 15-48-07.png}}
	% \end{picture}
\end{frame}
\begin{frame}{Robots are -\emph{almost}- everywhere !}	
	\begin{picture}(0,0)
		\put(0,70){Robots are able to accomplish complex missions...}
		\put(90,-20){\begin{minipage}{0.5\columnwidth}
				\centering
				% \href{run:Figures/Atlas Gets a Grip _ Boston Dynamics.mp4}{
					% \includegraphics[height=\columnwidth]{Screenshot from 2025-06-11 12-55-26.png}} 
						\embedvideo*{\includegraphics[width=1.20\textwidth,height=5cm]{Screenshot from 2025-06-11 12-55-26.png}}{Figures/Atlas Gets a Grip _ Boston Dynamics.mp4}[autoplay=false,showGUI=true]\\
				\makebox{\hspace{1.5cm}\centering\small \emph{Atlas} humanoid robot assisting workers}
		\end{minipage}}
	\end{picture}
\end{frame}
\begin{frame}{Robots are -\emph{almost}- everywhere !}	
	\begin{picture}(0,0)
		\put(0,70){... can cooperate ...}
		\put(90,-20){\begin{minipage}{0.5\columnwidth}
				\centering
				% \href{run:Figures/CooperativeLocomotion.m4v}{
				% 	\includegraphics[height=\columnwidth]{Screenshot from 2025-06-11 13-52-35.png}} \\
									\embedvideo*{\includegraphics[width=1.20\textwidth,height=5cm]{Screenshot from 2025-06-11 13-52-35.png}}{Figures/CooperativeLocomotion.m4v}[autoplay=false,showGUI=true]\\
				\makebox{\hspace{-0.8cm}\centering\small Two \emph{Go 1} quadruped robots cooperating to lift heavy objects}
		\end{minipage}}
	\end{picture}
\end{frame}
\begin{frame}[t]\frametitle{Robots are -\emph{almost}- everywhere !}
	\begin{picture}(0,0)
		\put(0,10){... and even replace humans in tedious tasks}
		\put(70,-90){\begin{minipage}{0.5\columnwidth}
				\centering
				% \href{run:Figures/CooperativeLocomotion.m4v}{
				% 	\includegraphics[height=\columnwidth]{Screenshot from 2025-06-11 13-52-35.png}} \\
			\embedvideo*{\includegraphics[width=1.3\textwidth,height=5.5cm]{Screenshot from 2025-06-11 13-52-35.png}}{Figures/Motivation-IAM.mp4}[autoplay=false,showGUI=true]\\
				\makebox{\hspace{1.8cm}\centering\small Results from the European Project I.AM}
		\end{minipage}}
	\end{picture}

\end{frame}
\begin{frame}{Robots are -\emph{almost}- everywhere !}	
	\begin{picture}(0,0)
		\put(0,70){... but have to satisfy some hard constraints}
		\put(00,-20){\begin{minipage}{0.5\columnwidth}
				\centering
				% \href{run:Figures/Control Barrier Function based safe navigation -- experiment.mp4}{
				% 	\includegraphics[height=\columnwidth]{Screenshot from 2025-06-11 16-33-50.png}} \\
										\embedvideo*{\includegraphics[width=\columnwidth,height=4.3cm]{Screenshot from 2025-06-11 16-33-50.png}}{Figures/Control Barrier Function based safe navigation -- experiment.mp4}[autoplay=false,showGUI=true]\\

				\makebox{\hspace{-0.2cm}\centering\small Getting to the target while avoiding collision}
		\end{minipage}}
		\put(220,-20){\begin{minipage}{0.5\columnwidth}
				\centering
				% \href{run:Figures/Safe Behavior through Control Barrier Functions.mp4}{
				% 	\includegraphics[height=\columnwidth]{Screenshot from 2025-06-11 16-33-28.png}} \\
					\embedvideo*{\includegraphics[width=\columnwidth,height=4.3cm]{Screenshot from 2025-06-11 16-33-28.png}}{Figures/Safe Behavior through Control Barrier Functions.mp4}[autoplay=false,showGUI=true]\\

				\makebox{\hspace{-0.5cm}\centering\small Keeping upright position}
		\end{minipage}}
	\end{picture}
\end{frame}
%\begin{frame}{Motivation - One control framework, many applications}
%	\vspace{-0.5cm}
%%	\begin{multicols}{2}
%		% Left column
%		\begin{tcolorbox}[colback=gray!5!white, colframe=gray!75!black, title=One robotic application]
%			\begin{itemize}\small
%				\item Use high DoF robots
%				\item Achieve complex maneuvers
%				\item Rely on sensors
%				\item Satisfy constraints: safety specifications, hardware limits, etc.
%			\end{itemize}
%		\end{tcolorbox}
%		%  \columnbreak $\overset{\rm Objective}{\longrightarrow}$
%		
%		% Right column
%	%	\columnbreak
%		\vspace{-0.5cm}
%		\begin{tcolorbox}[colback=green!5!white, colframe=green!70!black, title=Objective]
%			%	\textbf{Control framework}
%		\small	Is it possible to have \textbf{‘one control framework’} that can be used for \textbf{‘many robotic applications’}?
%		\end{tcolorbox}		
%%	\end{multicols}
%\end{frame}
%\begin{frame}{Motivation - One control framework, many applications}
%	The control framework should be: 
%%	\begin{multicols}{1}
%		% Left column		
%		\begin{tcolorbox}[colback=green!5!white, colframe=green!70!black]%-è, title=Need for a control framework]
%		%	\textbf{Control framework}
%			\begin{itemize} 
%				\item \textbf{Consistent} enough to render the required application complexity
%				\item \textbf{General} enough to encompass new applications
%				\item \textbf{Simple} enough for the user
%			\end{itemize}
%		\end{tcolorbox}		
%%	\end{multicols}
%\end{frame}
%
%\begin{frame}{Motivation - Control framework construction}
%	\begin{figure}%[t!]
%		\centering
%		\begin{tikzpicture}[auto, node distance=2cm,>=latex']
%			\node  [block](CRA) at (-1, 12) {Complex robotic application};	
%			\node (COP) at (2.2, 9) {Control objective primitives};
%			\node [block] (TC) at (2.2, 8)  {\begin{varwidth}{0.24\linewidth}
%					\begin{itemize}
%						\item Tasks
%						\item Constraints
%					\end{itemize}		
%			\end{varwidth}};
%			\node (hardware) at (7.4, 9) {Hardware};
%			\node [block] (RS) at (7.4, 8) {\begin{varwidth}{0.2\linewidth}
%					\begin{itemize}
%						\item Robots
%						\item Sensors
%					\end{itemize}	
%			\end{varwidth}};
%			\node (PT) at (-3.5, 9) {Parameters tuning};
%			\node [block] (PC) at (-3.5, 8) {\begin{varwidth}{0.31\linewidth}
%					\begin{itemize}
%						\item Priority
%						\item Convergence rate
%					\end{itemize}	
%			\end{varwidth}};
%			\node (w1) at (4.8, 7) {\small Atomic decomposition};
%			\node (w2) at (-0.6, 7) {\small Combination};
%			\node [tmp] at (-1.05, 12) (tmp) {}; 
%		
%			\draw [->] ([yshift=0pt]RS.west) -- node[below,pos=0.54]{}(TC);
%			\draw [->] ([yshift=0pt]TC.west) -- node[below,pos=0.54]{}(PC);
%		%	\draw [->] ([xshift=-60pt]PC.north) |- node[above,pos=0.7]{\small Complexity rendering}(tmp);
%			
%		\end{tikzpicture}
%	\end{figure}	
%\end{frame}
%\begin{frame}{Motivation - Control framework construction}
%	\begin{figure}
%		\centering
%		\begin{tikzpicture}[auto, node distance=2cm,>=latex']
%			% Nodes
%			\node [block] (CRA) at (-1, 12) {Complex robotic application};
%		%	\node [sum] (comp) at (4,11) {}; 
%			\node [block] (Sim) at (-1,10.5) {Simulation};
%			\node (COP) at (2.2, 9) {Control objective primitives};
%		%	\node [block] (adjust) at (6,11) {Adjustment};
%			\node [block] (TC) at (2.2, 8) {\begin{varwidth}{0.24\linewidth}
%					\begin{itemize}
%						\item Tasks
%						\item Constraints
%					\end{itemize}		
%			\end{varwidth}};
%			\node (hardware) at (7.4, 9) {Hardware};
%			\node [block] (RS) at (7.4, 8) {\begin{varwidth}{0.2\linewidth}
%					\begin{itemize}
%						\item Robots
%						\item Sensors
%					\end{itemize}	
%			\end{varwidth}};
%			\node (PT) at (-3.5, 9) {Parameters tuning};
%			\node [block] (PC) at (-3.5, 8) {\begin{varwidth}{0.31\linewidth}
%					\begin{itemize}
%						\item Priority
%						\item Convergence rate
%					\end{itemize}	
%			\end{varwidth}};
%			\node (w1) at (4.8, 7) {\small Atomic decomposition};
%			\node (w2) at (-0.6, 7) {\small Combination};
%			\node [tmp] at (-2.3, 10.5) (tmp) {}; 
%%			\node [tmp] at (7,10.5) (tmp2) {};
%		%	\node (w3) at (3, 10.2) {\small Complexity rendering};
%		%	\node (w4) at (4, 12.2) {\small Desired complexity};
%			% Existing arrows
%			\draw [->] (RS.west) -- (TC.east);
%			\draw [->] (TC.west) -- (PC.east);
%			\draw [->] ([xshift=-60pt]PC.north) |- (tmp);
%		%	\draw [->] (Sim.east) -| (comp.south);
%		%	\draw [->] (CRA.east) -| (comp.north);
%		%	\draw [->] (comp.east) -- (adjust.west);
%			% NEW: diagonal tuning arrows above blocks
%%			\foreach \n in {TC,RS,PC}{
%%				\draw[->,line width=2.5pt,red,opacity=0.55, dashed]  ([yshift=5pt]\n.north east) -- ([yshift=-5pt]\n.south west);
%%			}
%		\end{tikzpicture}
%	\end{figure}	
%\end{frame}
%\begin{frame}{Motivation - Control framework construction}
%	\begin{figure}
%		\centering
%		\begin{tikzpicture}[auto, node distance=2cm,>=latex']
%			% Nodes
%			\node [block] (CRA) at (-1, 12) {Complex robotic application};
%			\node [sum] (comp) at (4,11) {}; 
%			\node [block] (adjust) at (6,11) {\color{red}Adjustment};
%			\node [block] (Sim) at (-1,10.5) {Simulation};
%			\node (COP) at (2.2, 9) {Control objective primitives};
%			\node [block] (TC) at (2.2, 8) {\begin{varwidth}{0.24\linewidth}
%					\begin{itemize}
%						\item Tasks
%						\item Constraints
%					\end{itemize}		
%			\end{varwidth}};
%			\node (hardware) at (7.4, 9) {Hardware};
%			\node [block] (RS) at (7.4, 8) {\begin{varwidth}{0.2\linewidth}
%					\begin{itemize}
%						\item Robots
%						\item Sensors
%					\end{itemize}	
%			\end{varwidth}};
%			\node (PT) at (-3.5, 9) {Parameters tuning};
%			\node [block] (PC) at (-3.5, 8) {\begin{varwidth}{0.31\linewidth}
%					\begin{itemize}
%						\item Priority
%						\item Convergence rate
%					\end{itemize}	
%			\end{varwidth}};
%			\node (w1) at (4.8, 7) {\small Atomic decomposition};
%			\node (w2) at (-0.6, 7) {\small Combination};
%			\node [tmp] at (-2.3, 10.5) (tmp) {}; 
%			%			\node [tmp] at (7,10.5) (tmp2) {};
%			\node (w3) at (3, 10.2) {\small Rendered Complexity};
%			\node (w4) at (4, 12.2) {\small Desired complexity};
%			% Existing arrows
%			\draw [->] (RS.west) -- (TC.east);
%			\draw [->] (TC.west) -- (PC.east);
%			\draw [->] ([xshift=-60pt]PC.north) |- (tmp);
%			\draw [->] (Sim.east) -| (comp.south);
%			\draw [->] (CRA.east) -| (comp.north);
%			\draw [->] (comp.east) -- (adjust.west);
%			% NEW: diagonal tuning arrows above blocks
%			\foreach \n in {TC,RS,PC}{
%				\draw[->,line width=2.5pt,red,opacity=0.55, dashed]  ([yshift=5pt]\n.north east) -- ([yshift=-5pt]\n.south west);
%			}
%		\end{tikzpicture}
%	\end{figure}	
%\end{frame}
%\begin{frame}{Motivation - Control framework construction}
%	Example: Moving the base of a quadruped robot
%	\begin{minipage}{0.51\columnwidth}
%		\begin{figure}
%			\includegraphics[width=\columnwidth]{Screenshot from 2025-06-10 16-43-33}
%		\end{figure}
%	\end{minipage}
%	\hfill
%	\begin{minipage}{0.48\columnwidth}
%		 \invisible{
%		\begin{itemize}
%			\item \textbf{Task:} move the base in the manipulability space
%			\item \textbf{Constraints:} 
%			\begin{itemize}
%				\item Keep end-effector fixed
%				\item Keep non-slipping contacts
%				\item Avoid self-collision
%				\item Respect joint position/velocity/torque constraints
%				\item Keep CoM projection inside the static equilibrium region
%			\end{itemize}
%		\end{itemize}}
%	\end{minipage}
%\end{frame}
%\begin{frame}{Motivation - Control framework construction}
%	Example: Moving the base of a quadruped robot
%	\begin{minipage}{0.51\columnwidth}
%		\begin{figure}
%			\includegraphics[width=\columnwidth]{multi-task example.pdf}
%		\end{figure}
%	\end{minipage}
%	\hfill
%	\begin{minipage}{0.48\columnwidth}
%		\begin{itemize}
%			\item \textbf{Task:} move the base in the manipulability space
%			\item \textbf{Constraints:} 
%			\begin{itemize}
%				\item Keep end-effector fixed
%				\item Keep non-slipping contacts
%				\item Avoid self-collision
%				\item Respect joint position/velocity/torque constraints
%				\item Keep CoM projection inside the static equilibrium region
%			\end{itemize}
%		\end{itemize}
%		
%	\end{minipage}
%\end{frame}
%\begin{frame}{Motivation - Control framework construction}
%\textbf{	One \emph{elegant} solution: Quadratic Programming paradigm }
%	\begin{align*}
%		\left.\begin{matrix}
%			\textit{Multi}\text{-robot} \\
%			\textit{Multi}\text{-task} \\
%			\textit{Multi}\text{-sensor}
%		\end{matrix}
%		\right\}		
%		\text{Task-space Quadratic Programming Control}
%	\end{align*}
%\end{frame}
%\begin{frame}{QP Control - Historical overview}
%	%\begin{columns}
%		%\column{0.5\linewidth}
%		\begin{itemize}
%			\item  \textbf{1960s:} Most works focused on numerical algorithms for QP solving
%			\item \textbf{1970s:} Emergence of Model Predictive Control (MPC) - Use of QP as a \emph{tool} for solving the MPC problem~\cite{richalet1978automatica}
%			\item \textbf{1980s:} First use QP for trajectory planning with collision avoidance for redundant manipulator control as an alternative to artificial potential fields method~\cite{faverjon1987icra}
%			\item \textbf{1990s:} Premises of QP as a \emph{‘controller’} - Standard QP formulations are posed for redundant manipulator control with joints constraints~\cite{cheng1994tra,park1998icra}
%		\end{itemize}
%\end{frame}
%\begin{frame}{QP Control - Historical overview}
%		\begin{itemize}
%			
%			\item \textbf{(2000-2010)s:} Multi-objective control - Use of QP for motion generation in graphical animations of robots in multi-contact~\cite{zhang2004transactionsonSysManCyb2,abe2007siggraph,macchietto2009siggraph,salini2010springer} 
%			\item \textbf{(2010-Now):} Golden era of QP control - Real-time QP control of large classes of robots (manipulators, humanoids, quadrupeds, drones, unicycles, etc.) with real world application - New control strategy enabled - Breakthrough theoretical contributions 
%			\item \textbf{(2021-Now):} Real-time implementation of whole-body MPC 
%		\end{itemize}
%		%\column{0.4\linewidth}
%% 		\begin{tikzpicture}[spy using outlines={rectangle,thick,red,fill=red,opacity=0.2,
%  %                     lens={scale=2.5}, width=2.5cm, height=1cm, connect spies}]
%  %   % Place main figure
%  %   \node (image) {\includegraphics[width=0.5\columnwidth]{Figures/mpc_history_0.png}};
%  %   % Spy on coordinates (x,y) in the image coordinate system
%  %   % Adjust (x,y) to your region of interest
%  %   \spy on (0.12,0.7) in node [fill=white] at (4,1);
%  %   % \fill[red,opacity=0.2] (0.41,0.8) rectangle (0.59,0.70);
%  % \end{tikzpicture}
%% 	\end{columns}
%\end{frame}
\begin{frame}{Outline}
	\tableofcontents
\end{frame}
%\begin{frame}{Motivation - Control framework construction}
%One solution: \textbf{Quadratic Programming} paradigm 
%	\begin{align*}
%		\begin{matrix}
%			\textit{Multi-robots} \\
%			\textit{Multi-tasks} \\
%			\textit{Multi-sensors}
%		\end{matrix}
%		\left\}\begin{matrix}
%		\begin{split}
%			\underset{\decisionVar}{\min}&\frac{1}{2}\decisionVar^T\mathbf{H}\decisionVar +\bm{h}^T\decisionVar \\
%			\rm{S.t:~}&\mathbf{C}_{\rm ineq}\decisionVar\leq\bm{d}_{\rm ineq}\\
%			&\mathbf{C}_{\rm eq}\decisionVar=\bm{d}_{\rm eq}
%		\end{split}
%	\end{matrix}\right.
%	\end{align*}
%\end{frame}	
\section{Kinematic Modeling}
\begin{frame}{Rotation matrix }
	\begin{figure}
		\centering
		\includegraphics[width=0.5\columnwidth]{2frames.pdf}
	\end{figure}
	\begin{itemize}
		\item 	The orientation of the frame $\setR_B$ w.r.t the frame $\setR_I$ can be represented by the rotation matrix $\matriceR_B^I=\begin{bmatrix} \vectorX_B & \vectorY_B& \vectorZ_B\end{bmatrix}\in\rm{SO}(3)$
		\item $\vectorX_B, \vectorY_B,\vectorZ_B\inR^3$ are expressed in $\setR_I$
	\end{itemize}

\end{frame}
\begin{frame}{Rotation matrix }
	\begin{itemize}
		\item The columns of $\matriceR_B^I$ are orthogonal to each other: $\vectorX_B^T\vectorY_B=0$, $\vectorY_B^T\vectorZ_B=0$, $\vectorX_B^T\vectorZ_B=0$
		\item The columns of $\matriceR_B^I$ are normal: $\vectorX_B^T\vectorX_B=1$, $\vectorY_B^T\vectorY_B=1$, $\vectorZ_B^T\vectorZ_B=1$
		\item$\matriceR_I^B=\left(\matriceR_B^I\right)^{-1}=\left(\matriceR_B^I\right)^T$
		\item $\rm{SO}(3)$ denotes the group of $3\times3$ orthogonal matrices $\matriceR$ such that $\matriceR^T = \matriceR$ and $\det(\matriceR)=1$
		\item Sucessive rotations are expressed by rotation composition: $\matriceR_C^I = \matriceR_B^I\matriceR_C^B$. Meaning: perfirm a rotation $\matriceR_B^I$ w.r.t $\setR_I$, then perform a rotation $\matriceR_C^B$ w.r.t $\setR_B$
		\item \textbf{Remark:} rotation composition is generally not commutative!
	\end{itemize}
\end{frame}
\begin{frame}{Rotation matrix }
	\begin{itemize}
		\item If $\vectorV^B\inR^3$ is expressed in the frame $\setR_B$, then it can be expressed in the frame $\setR_I$ by $\vectorV=\matriceR_B^I\vectorV^B$
		\item Rotation matrix time derivative: 
		\begin{align*}
			\matriceRdot_B^I&=\matriceR_B^I\left[\vectorOmega_B^B\times\right]\\
			\matriceRdot_B^I&=\left[\vectorOmega_B^I\times\right]\matriceR_B^I
		\end{align*} 
		\item Orientation can also be represented by: Euler angles, Angle-axis, Quaternions
	\end{itemize}
\end{frame}
\begin{frame}{Rigid body kinematics}
	A rigid body kinematics is completely described by the \emph{position} and \emph{orientation} (\emph{pose}) of the frame $\setR_B$ (attached to the rigid body) w.r.t a reference frame $\setR_I$. 
	\begin{figure}
		\centering
		\includegraphics[width=0.6\columnwidth]{rigid-body.pdf}
	\end{figure}
\end{frame}
\begin{frame}{Rigid body kinematics}
The rigid body position is given by $\boldsymbol{o}_B\inR^3$ (expressed in $\setR_I$), whereas the orientation can be defined by the rotation matrix $\matriceR_B^I$%=\begin{bmatrix} \vectorX_B & \vectorY_B& \vectorZ_B\end{bmatrix}\in \rm{SO}(3)$
	\begin{figure}
		\centering
		\includegraphics[width=0.5\columnwidth]{rigid-body.pdf}
	\end{figure}
\end{frame}

\begin{frame}{Rigid body kinematics}
	The rigid body pose can be compactly represented by a homogeneous transformation matrix 
	\begin{equation*}
		\matriceT_B^I=\begin{bmatrix}
			\matriceR_B^I & \vectorO_B \\ \boldsymbol{0}_{1\times 3} &1
		\end{bmatrix}\in {\rm SE}(3)
	\end{equation*}
	
	\begin{figure}
		\centering
		\includegraphics[width=0.5\columnwidth]{rigid-body.pdf}
	\end{figure}
\end{frame}
\begin{frame}{Homegeneous transformation}
%	\textbf{Note:}
	\begin{itemize}
		\item The homogeneous transformation matrix performs a translation and a rotation of a rigid body without changing its dimensions while preserving the right handedness 
		\item Case of a pure translation:
		\begin{equation*}
			\matriceT_B^I=\begin{bmatrix}
				\matriceI_3 & \vectorO_B \\ \boldsymbol{0}_{1\times 3} &1
			\end{bmatrix}
		\end{equation*} 
		\item Case of a pure rotation:  
		\begin{equation*}
			\matriceT_B^I=\begin{bmatrix}
				\matriceR_B^I  & \boldsymbol{0}_{3\times 1}  \\ \boldsymbol{0}_{1\times 3} &1
			\end{bmatrix}
		\end{equation*}
	\end{itemize}
\end{frame}
\begin{frame}{Homegeneous transformation}
	The inverse of the homogeneous transformation matrix is given by 
	\begin{align*}
		\left(\matriceT_B^I\right)^{-1} = \matriceT^B_I &= \begin{bmatrix}
			\left(\matriceR_B^I\right)^T & -\left(\matriceR_B^I\right)^T \vectorO_B\\
			\vectorZero_{1\times3} &1
		\end{bmatrix} \\
		&=\begin{bmatrix}
			\matriceR^B_I & -\matriceR^B_I \vectorO_B\\
			\vectorZero_{1\times3} &1
		\end{bmatrix}
	\end{align*} 
\end{frame}
\begin{frame}{Homegeneous transformation}
	In homogeneous coordinates, a point is represented by 
	\begin{equation*}
		\underline{\vectorP} = \begin{bmatrix}
			x\\y\\z\\1
		\end{bmatrix}
	\end{equation*}
	while a velocity vector is represented as 
	\begin{equation*}
		\underline{\vectorV} = \begin{bmatrix}
			v_x\\v_y\\v_z\\0
		\end{bmatrix}
	\end{equation*}
\end{frame}
\begin{frame}{Homegeneous transformation}
	Given a point $P$ with homogeneous coordinates $\underline{\vectorP}_B$ expressed in the frame $\setR_B$. Its coordinates $\underline{\vectorP}$ expressed in $\setR_I$ can be computed as 
	\begin{align*}
		\underline{\vectorP} &= \matriceT_B^I\underline{\vectorP}_B 
	\end{align*}
	\begin{figure}
		\centering
		\includegraphics[width=0.5\columnwidth]{TransformationMatrix-PositionVector}
	\end{figure}
\end{frame}
\begin{frame}{Homegeneous transformation}
	Given a velocity vector with homogeneous coordinates $\underline{\vectorV}_B$ expressed in $\setR_B$. Its coordinates $\underline{\vectorV}$ expressed in $\setR_I$ can be computed as 
	\begin{align*}
		\underline{\vectorV} &= \matriceT_B^I\underline{\vectorV}_B 
	\end{align*}
	\begin{figure}
		\centering
		\includegraphics[width=0.55\columnwidth]{TransformationMatrix-VelocityVector}
	\end{figure}
\end{frame}
%\begin{frame}{Homegeneous transformation}
%	\begin{itemize}	
%		\item The position and orientation of the rigid body w.r.t the frame $\setR_I$ can be completely described by $\matriceT_B^I$: 
%		\begin{align*}
%			\begin{bmatrix}
%				\vectorO_B \\ 1
%			\end{bmatrix} &= \matriceT_B^I\begin{bmatrix}
%				0\\0\\0 \\ 1
%			\end{bmatrix}\\
%			\matriceR_B^I &= \matriceT_B^I(1:3,1:3)
%		\end{align*}
%	\end{itemize}
%\end{frame}
%\begin{frame}{Homegeneous transformation}
%	Consider a point $P$ on the rigid body. The position $\boldsymbol{p}_I$ of $P$ w.r.t the frame $\setR_I$ can be expressed  by the homogeneous transformation 
%	\begin{equation*}
%		{\color{blue}\vectorP_I} = {\color{red}\vectorO_B} + \matriceR_B^I{\color{SeaGreen}\boldsymbol{p}_B}
%	\end{equation*}
%	where $\boldsymbol{p}_B$ is the position of the point $P$ in the frame $\setR_B$
%		\begin{figure}
%		\centering
%		\includegraphics[width=0.5\columnwidth]{rigid-body_TransformationMatrix.pdf}
%	\end{figure}
%\end{frame}
%\begin{frame}{Homegeneous transformation}
%Alternatively, the position $\vectorP_I$ can be computed by the homegeneous transformation matrix $\matriceT_B^I$
%	\begin{equation*}
%		\begin{bmatrix}
%			\boldsymbol{p}_I \\ 1
%		\end{bmatrix} = \matriceT_B^I\begin{bmatrix}
%			\boldsymbol{p}_B \\ 1
%		\end{bmatrix}, 
%	\end{equation*}
%	\begin{figure}
%		\centering
%		\includegraphics[width=0.5\columnwidth]{rigid-body_TransformationMatrix.pdf}
%	\end{figure}
%\end{frame}
\begin{frame}{Homegeneous transformation}
The {\color{blue}{position}} and {\color{purple}orientation} of the  frame $\setR_C$ w.r.t to $\setR_I$ can be computed through the homogeneous transformation matrix composition
			
	\begin{minipage}{0.3\columnwidth}
		\begin{align*}
			\matriceT_C^I&=\matriceT_B^I\matriceT_C^B\\
			&=\begin{bmatrix}
				\matriceR_B^I&{\color{red}\vectorO_B}\\\vectorZero_{1\times3}&1
			\end{bmatrix}\begin{bmatrix}
				\matriceR_C^B&{\color{SeaGreen}\vectorO_C^B}\\\vectorZero_{1\times3}&1
			\end{bmatrix} \\
			&=\begin{bmatrix}
				\matriceR_B^I\matriceR_C^B&{\color{blue}\vectorO_B+\matriceR_B^I\vectorO_C^B}\\\vectorZero_{1\times3}&1
			\end{bmatrix}\\
			&=\begin{bmatrix}
				{\color{purple}\matriceR_C^I}&{\color{blue}\vectorO_B+\matriceR_B^I\vectorO_C^B}\\\vectorZero_{1\times3}&1
			\end{bmatrix}
		\end{align*}
	\end{minipage}
	\hfill
	\begin{minipage}{0.4\columnwidth}
		\begin{figure}
			\centering
			\includegraphics[width=\columnwidth]{rigid-body_TransformationMatrix_rot.pdf}
		\end{figure}
	\end{minipage}
\end{frame}
\begin{frame}{Homegeneous transformation}
Following the same reasoning, the position and orientation of the end-effector frame  $\setR_{ee}$ w.r.t to $\setR_I$ can be obtained through the homogeneous transformation matrix composition
	
	\begin{minipage}{0.2\columnwidth}
		\begin{align*}
			\matriceT_{ee}^I&=\matriceT_B^I\matriceT_C^B\matriceT_D^C\matriceT_E^D\matriceT_F^E\matriceT_{ee}^{F}\\
		\end{align*}
	\end{minipage}
	\hfill
	\begin{minipage}{0.6\columnwidth}
		\begin{figure}
			\centering
			\includegraphics[height=0.4\textheight]{rigid-tree_TransformationMatrix.pdf}
		\end{figure}
	\end{minipage}
\end{frame}
\begin{frame}{Homegeneous transformation}
\begin{itemize}
	\item The homogeneous transformation matrix $\matriceT_{ee}^I$ depends on the choice of the frames placement
	\item Need for a \textbf{standard method} to decribe the relative transformation matrices between each pair of bodies with a \textbf{minimum number of parameters}\footnote[frame]{A homogeneous transformation matrix needs 9 parameters.}
\end{itemize}
	
	\begin{minipage}{0.2\columnwidth}
		\begin{align*}
			\matriceT_{ee}^I&=\begin{bmatrix}
				{\matriceR_{ee}^I}&{\vectorP_{ee}}\\\vectorZero_{1\times3}&1
			\end{bmatrix}
		\end{align*}
	\end{minipage}
\end{frame}
\begin{frame}{Denavit-Hartenberg method}
DH convention is based on the following rules for the frames placement
\begin{itemize}
	\item Choose axis $z_j$ along the axis of joint $j+1$
	\item Choose axis $x_j$ such that $x_j\perp z_{j-1}$ and intersects with it
\end{itemize}
\begin{figure}
	\centering
	\includegraphics[height=0.55\textheight]{bodies&joints.pdf}
\end{figure}
\end{frame}
\begin{frame}{Denavit-Hartenberg method}
The homogeneous transformation matrix $\matriceT_{j}^{j-1}$ is defined using 4 parameters

	\begin{minipage}{0.5\columnwidth}
		\begin{itemize}
			\item $\theta_j$: angle between $x_{j-1}$ and $x_j$ around $z_{j-1}$
			\item $d_j$: distance between $x_{j-1}$ and $x_j$ along $z_{j-1}$
			\item $\alpha_j$: angle between $z_{j-1}$ and $z_j$ around $x_{j}$
			\item $a_j$: distance between $z_{j-1}$ and $z_j$ along $x_{j}$
		\end{itemize}
	\end{minipage}
	\hfill
	\begin{minipage}{0.45\columnwidth}
			\begin{figure}
			\centering
			\includegraphics[width=\columnwidth]{DH.pdf}
		\end{figure}
	\end{minipage}
%	$a_j$ and $\alpha_j$ are always constant and depends on the placement of the joint on the robot structure, whereas $\theta_j$ and $d_j$ depends on the type of the joint:
%	\begin{itemize}
%		\item If joint $j$ is revolute $ \Rightarrow\theta_j$ is variable, $d_j$ constant
%		\item If joint $j$ is prismatic $ \Rightarrow d_j$ is variable, $\theta_j$ constant
%	\end{itemize} 
	

\end{frame}
\begin{frame}{Denavit-Hartenberg method}
	\begin{itemize}
		\item $a_j$ and $\alpha_j$ are \textbf{always constant} and depend on the placement of the joint on the robot structure,
		\item $\theta_j$ and $d_j$ depends on the type of the joint:
		\begin{itemize}
			\item If joint $j$ is \textbf{revolute} $ \Rightarrow\theta_j$ is variable, $d_j$ constant
			\item If joint $j$ is \textbf{prismatic} $ \Rightarrow d_j$ is variable, $\theta_j$ constant
		\end{itemize} 
		\item The homogeneous transformation matrix $\matriceT_{j}^{j-1}$ is given as 
		\begin{equation*}
			\matriceT_{j}^{j-1} = \begin{bmatrix}
				\cos(\theta_j)&-\sin(\theta_j)\cos(\alpha_j) &\sin(\theta_j)\sin(\alpha_j) & a_j\cos(\theta_j)\\
				\sin(\theta_j)&\cos(\theta_j)\cos(\alpha_j) &-\cos(\theta_j)\sin(\alpha_j) & a_j\sin(\theta_j)\\
				0&\sin(\alpha_j)&\cos(\alpha_j)&d_j\\
				0&0&0&1
			\end{bmatrix}
		\end{equation*}
	\end{itemize} 	
\end{frame}	
%\section{Modélisation Cinématique}
%%\subsection{Handling Inequality Constraints}
%\begin{frame}
%\centering
%\Large Handling Inequality Constraints
%\end{frame}
\begin{frame}{QP Control - Introductory example}
Mass sliding on a surface

\begin{minipage}{0.4\columnwidth}
	\begin{tikzpicture}[>=latex', scale=1]

% Parameters (change angle or size here)
\def\planeAngle{0}   % angle of the plane in degrees
\def\s{.8}           % half-size of the square (so square side = 2*\s)
\def\gap{0.39}        % gap between square and plane for clarity
\def\arrowlen{\s}    % length scaling for force arrows

% Coordinates: base point on plane (origin)
\coordinate (O) at (0,0);
% point on plane to draw a long line
\coordinate (P) at (6,0);
% rotate coordinate system by planeAngle to place plane horizontally in rotated frame
\begin{scope}[rotate=\planeAngle]

  % draw plane (a long rectangle strip)
  \fill[gray!20] (-0.5,-0.4) -- (5,-0.4) -- (5,0.4) -- (-0.5,0.4) -- cycle;
  \draw[black,line width=0.6pt] (-1,0) -- (6,0);

  % place square center at x = 2.2 along rotated axis
  \coordinate (C) at (2.2, \s + \gap);

  % draw square (centered at C)
  \draw[fill=blue!5] ($(C)+(-\s,-\s)$) rectangle ++(2*\s,2*\s);
  \draw ($(C)+(0,0)$) node {$m$};

  % draw velocity vector (down the plane)
  %\draw[->, line width=1pt] ($(C)+(0,-\s)$) -- ++(0,-\arrowlen*0.55) node[midway,right] {$\mathbf{v}$};

  % gravity at center: draw downward in world frame (rotate back)
\end{scope}

% Transform C back to world coords by rotating
\coordinate (Cw) at ($(C)$);
\coordinate (Cworld) at ($(Cw)$);
% Actually the above C was defined in rotated scope; easier: recompute Cworld
\coordinate (Cworld) at ($( {2.2*cos(\planeAngle) - ( \s+\gap)*sin(\planeAngle) } , {2.2*sin(\planeAngle) + ( \s+\gap)*cos(\planeAngle) } )$);

% draw gravity (mg) downward (world vertical)
%\draw[->, line width=1pt] (Cworld) -- ++(0,-\arrowlen) node[below] {$m\mathbf{g}$};

% draw normal N (perpendicular to plane) at contact point
% contact point on plane under the cube (project center to plane)
\coordinate (proj) at ($(Cworld) + ({( \s+\gap)*sin(\planeAngle)},{-(\s+\gap)*cos(\planeAngle)})$);
\coordinate (applicationPoint) at (2.2 - 2*\s, \s + \gap);
% Normal direction (unit vector perpendicular to plane, pointing away from plane)
%\draw[->, line width=1pt] (proj) -- ++({\arrowlen*cos(\planeAngle)},{\arrowlen*sin(\planeAngle)}) node[right] {$\mathbf{N}$};

% Friction direction (parallel to plane, up the plane if block slides down)
\draw[->, line width=1pt] (applicationPoint) -- ++({ \arrowlen*cos(\planeAngle)},{ \arrowlen*sin(\planeAngle)}) node[above left] {${u}$};

% Draw components of gravity relative to plane: g_parallel and g_perp (dashed)
% Vector from center along negative plane-parallel direction
%\draw[dashed] (Cworld) -- ++({\arrowlen*sin(\planeAngle)},{-\arrowlen*cos(\planeAngle)}) coordinate (gpar);
%\draw[->, dashed] (Cworld) -- ++({\arrowlen*cos(90-\planeAngle)*0.9*sin(1)},{- \arrowlen}); % no-op to show dashed style (kept for clarity)

% Instead compute exact components using trig:
% g_parallel: along plane downward (points down the slope)
%\draw[->, blue!60!black, line width=0.9pt] (Cworld) -- ++({\arrowlen*sin(\planeAngle)},{- \arrowlen*cos(\planeAngle)}) node[midway, below right] {$mg\sin\theta$};
% g_perp: perpendicular to plane (points into plane)
%\draw[->, red!70!black, line width=0.9pt] (Cworld) -- ++({-\arrowlen*cos(\planeAngle)},{- \arrowlen*sin(\planeAngle)}) node[midway, left] {$mg\cos\theta$};

% draw angle label theta at origin between horizontal and plane
%\draw[->] (0.8,0.18) arc (0:\planeAngle:0.8) node[right, yshift=2] {$\theta$};

% Axes for reference
\draw[->] (-0.5,0) -- (6,0) node[above] {$y$};
%\draw[-] (-0.5,-0.4) -- (5,-0.4) node[below] {$y_{\max}$};
%\draw[-] (5,-0.4) -- (-0.5,-0.4) node[below] {$y_{\min}$};
%\draw[->] (0,-1.2) -- (0,3.2) node[above] {$y$};

% small note
%\node[align=left, right] at (6.2,2.6) {Square mass sliding \\ on an inclined plane};

\end{tikzpicture}
\end{minipage}
\hfill
\begin{minipage}{0.55\columnwidth}
Equation of motion
	\begin{align*}
		\begin{split}
			m\ddot{y}&=u, \ \vectorX=\begin{bmatrix}
				 y-y_{\rm d} \\ \dot{y}
			\end{bmatrix}\\
			\bm{\dot{x}} &= \begin{bmatrix}
				0 & 1 \\ 0 & 0
			\end{bmatrix}\vectorX + \begin{bmatrix}
				0 \\\frac{1}{m}
			\end{bmatrix} u = f(\vectorX,u)
		\end{split}
	\end{align*}	
\end{minipage}
The goal is to steer the mass to a desired position $y_{\rm d}$. 
Let's assume there exists a feedback control $u=\phi_1(\vectorX)$ such that the origin of the closed-loop system $f(\vectorX,\phi_1(\vectorX))$ is asymptotically stable.
% while respecting 
% \begin{itemize}
% 		\item $u_{\min}\leq u\leq u_{\max}$
% \end{itemize}

%Let's assume there exists a feedback $u=\phi_1(\vectorX)$ such that the origin of the closed-loop system $f(\vectorX,\phi_1(\vectorX))$ is asymptotically stable.
% \begin{itemize}
% 	\item What is the context of QP use in control
% 	\item How QP can be used in control context? %A bit of history
% 	\item 
% \end{itemize}	
\end{frame}

\begin{frame}{QP Control - Introductory example}
\begin{itemize}
	\item Now, let's assume there exist constraints on the control input $u$
\begin{itemize}
		\item $u_{\min}\leq u\leq u_{\max}$
\end{itemize}
	\item How to account for these constraints in the control law synthesis?
	\item Solution: 
	\begin{equation*}
		u=\begin{matrix} \max\left(\min(\phi_1(\vectorX),u_{\max}),u_{\min}\right) \end{matrix},
	\end{equation*} 
	or equivalently 
	\begin{align*} 
		\begin{split}
	 		\underset{u}{\min}&\frac{1}{2}
	 		\begin{Vmatrix}
	 			u-\phi_1(\vectorX)
	 		\end{Vmatrix}^2\\
		\rm{S.t:~}&u_{\min}\leq u\leq u_{\max}\\
	 	\end{split}
	\end{align*}
	\item QP \textbf{outputs} the control law, and takes the system state as \textbf{input}
\end{itemize}
\end{frame}

\begin{frame}{Handling Inequality Constraints - Position Constraint}
Now, let's assume that the mass motion is constrained to remain within $y_{\min}$ et $y_{\max}$ 

\begin{minipage}{0.4\columnwidth}
	\begin{tikzpicture}[>=latex', scale=1]

% Parameters (change angle or size here)
\def\planeAngle{0}   % angle of the plane in degrees
\def\s{.8}           % half-size of the square (so square side = 2*\s)
\def\gap{0.39}        % gap between square and plane for clarity
\def\arrowlen{\s}    % length scaling for force arrows

% Coordinates: base point on plane (origin)
\coordinate (O) at (0,0);
% point on plane to draw a long line
\coordinate (P) at (6,0);
% rotate coordinate system by planeAngle to place plane horizontally in rotated frame
\begin{scope}[rotate=\planeAngle]

  % draw plane (a long rectangle strip)
  \fill[gray!20] (-0.5,-0.4) -- (5,-0.4) -- (5,0.4) -- (-0.5,0.4) -- cycle;
  \draw[black,line width=0.6pt] (-1,0) -- (6,0);

  % place square center at x = 2.2 along rotated axis
  \coordinate (C) at (2.2, \s + \gap);

  % draw square (centered at C)
  \draw[fill=blue!5] ($(C)+(-\s,-\s)$) rectangle ++(2*\s,2*\s);
  \draw ($(C)+(0,0)$) node {$m$};

  % draw velocity vector (down the plane)
  %\draw[->, line width=1pt] ($(C)+(0,-\s)$) -- ++(0,-\arrowlen*0.55) node[midway,right] {$\mathbf{v}$};

  % gravity at center: draw downward in world frame (rotate back)
\end{scope}

% Transform C back to world coords by rotating
\coordinate (Cw) at ($(C)$);
\coordinate (Cworld) at ($(Cw)$);
% Actually the above C was defined in rotated scope; easier: recompute Cworld
\coordinate (Cworld) at ($( {2.2*cos(\planeAngle) - ( \s+\gap)*sin(\planeAngle) } , {2.2*sin(\planeAngle) + ( \s+\gap)*cos(\planeAngle) } )$);

% draw gravity (mg) downward (world vertical)
%\draw[->, line width=1pt] (Cworld) -- ++(0,-\arrowlen) node[below] {$m\mathbf{g}$};

% draw normal N (perpendicular to plane) at contact point
% contact point on plane under the cube (project center to plane)
\coordinate (proj) at ($(Cworld) + ({( \s+\gap)*sin(\planeAngle)},{-(\s+\gap)*cos(\planeAngle)})$);
\coordinate (applicationPoint) at (2.2 - 2*\s, \s + \gap);
% Normal direction (unit vector perpendicular to plane, pointing away from plane)
%\draw[->, line width=1pt] (proj) -- ++({\arrowlen*cos(\planeAngle)},{\arrowlen*sin(\planeAngle)}) node[right] {$\mathbf{N}$};

% Friction direction (parallel to plane, up the plane if block slides down)
\draw[->, line width=1pt] (applicationPoint) -- ++({ \arrowlen*cos(\planeAngle)},{ \arrowlen*sin(\planeAngle)}) node[above left] {${u}$};

% Draw components of gravity relative to plane: g_parallel and g_perp (dashed)
% Vector from center along negative plane-parallel direction
%\draw[dashed] (Cworld) -- ++({\arrowlen*sin(\planeAngle)},{-\arrowlen*cos(\planeAngle)}) coordinate (gpar);
%\draw[->, dashed] (Cworld) -- ++({\arrowlen*cos(90-\planeAngle)*0.9*sin(1)},{- \arrowlen}); % no-op to show dashed style (kept for clarity)

% Instead compute exact components using trig:
% g_parallel: along plane downward (points down the slope)
%\draw[->, blue!60!black, line width=0.9pt] (Cworld) -- ++({\arrowlen*sin(\planeAngle)},{- \arrowlen*cos(\planeAngle)}) node[midway, below right] {$mg\sin\theta$};
% g_perp: perpendicular to plane (points into plane)
%\draw[->, red!70!black, line width=0.9pt] (Cworld) -- ++({-\arrowlen*cos(\planeAngle)},{- \arrowlen*sin(\planeAngle)}) node[midway, left] {$mg\cos\theta$};

% draw angle label theta at origin between horizontal and plane
%\draw[->] (0.8,0.18) arc (0:\planeAngle:0.8) node[right, yshift=2] {$\theta$};

% Axes for reference
\draw[->] (-0.5,0) -- (6,0) node[above] {$y$};
\draw[-, line width=2pt] (-0.5,-0.4) -- (5,-0.4) node[below] {$y_{\max}$};
\draw[-, line width=1pt] (5,-0.4) -- (-0.5,-0.4) node[below] {$y_{\min}$};
%\draw[->] (0,-1.2) -- (0,3.2) node[above] {$y$};

% small note
%\node[align=left, right] at (6.2,2.6) {Square mass sliding \\ on an inclined plane};

\end{tikzpicture}
\end{minipage}
%\hfill
%\begin{minipage}{0.55\columnwidth}
%How to account for the limit on $y$?
%	
%\end{minipage}

The goal is to steer the mass to a desired position $y_{\rm d}$ while respecting 
\begin{itemize}
	\item $u_{\min}\leq u\leq u_{\max}$, and 
	\item $y_{\min}\leq y\leq y_{\max}$
\end{itemize}
\end{frame}

\begin{frame}{Handling Inequality Constraints - Position Constraints}
\begin{itemize}
	\item In this case, $y$ does not appear in decision variables vector $\decisionVar$
	\item Need for more elaborated approach... 
	\end{itemize}
\end{frame}

\begin{frame}{Handling Inequality Constraints - Position Constraint}
	\hypertarget{position constraint}{}
Think in terms of acceleration\footnote[frame]{Same steps apply for $h={y} -{y}_{\min}$.}\bigskip
\hyperlink{CBF applications}{\beamerbutton{Back to CBF}}
	\begin{minipage}{0.4\columnwidth}
	\begin{align*}
		h&=y_{\max} -y\\
		\dot{h} &= -\dot{y}\\
		\ddot{h} &=-\ddot{y}\\
		\vectorX_h &= \begin{bmatrix}h \\ \dot{h}\end{bmatrix}
	\end{align*}
	\end{minipage}
	%\hfill
	%\begin{minipage}{0.55\columnwidth}
		
		Assume there exists $\phi_2(\vectorX_h)$ such that if $\ddot{h} = \phi_2(\vectorX_h)$ then $h(t)=\alpha e^{-\lambda t}$. 
		% Note that if $h(t_0)\geq0 \ \Rightarrow h(t)\geq0, \  \forall t\geq t_0 \ \Rightarrow {\color{red}y(t)\leq y_{\max}}$
	%\end{minipage}
\end{frame}

\begin{frame}{Handling Inequality Constraints - Position Constraint}
	Then, if 
	\begin{equation*}
	\ddot{h} \geq \phi_2(\vectorX_h)~\left(\Leftrightarrow\ddot{y}\leq-\phi(\vectorX_h)\right), \ \text{and~} h(t_0)\geq0~\left(\Leftrightarrow{y}(t_0)\leq y_{\max}\right)
	\end{equation*}
	by virtue of the Comparison Lemma\footnote[frame]{Differential inequality solution, see~\cite[Lemma~3.4]{khalil2002NonLinearSystems}.} 
	\begin{equation}
		h(t)\geq\alpha e^{-\lambda t}\geq0, \  \forall t\geq t_0 \ \Rightarrow {\color{red}y(t)\leq y_{\max}}
	\end{equation}
\end{frame}

\begin{frame}{Handling Inequality Constraints - Position Constraint}
QP is then formulated as 
\begin{align*}
	\begin{split}
	 		\underset{{\color{red}\decisionVar=\begin{bmatrix}u,\ddot{y}\end{bmatrix}}}{\min}&\frac{1}{2}
	 		\begin{Vmatrix}
	 			u-\phi_1(\vectorX)
	 		\end{Vmatrix}^2\\
		\rm{S.t:~}&u_{\min}\leq u\leq u_{\max}\\
							&\ddot{y}\leq-\phi_2(\vectorX_h)\\
							&{\color{red}m\ddot{y}=u}
	 	\end{split}
\end{align*}
\begin{itemize}
	\item The decision variables vector {\color{red}$\decisionVar$} is extended
	\item The system dynamics {\color{red}(equation of motion)} is accounted for to induce coupling between $\ddot{y}$ and $u$ 
\end{itemize}
\end{frame}
\begin{frame}{Handling Inequality Constraints - Velocity Constraint}
Now, let's assume that the mass velocity is constrained to remain within $\dot{y}_{\min}$ et $\dot{y}_{\max}$ 

\begin{minipage}{0.4\columnwidth}
	\begin{tikzpicture}[>=latex', scale=1]

% Parameters (change angle or size here)
\def\planeAngle{0}   % angle of the plane in degrees
\def\s{.8}           % half-size of the square (so square side = 2*\s)
\def\gap{0.39}        % gap between square and plane for clarity
\def\arrowlen{\s}    % length scaling for force arrows

% Coordinates: base point on plane (origin)
\coordinate (O) at (0,0);
% point on plane to draw a long line
\coordinate (P) at (6,0);
% rotate coordinate system by planeAngle to place plane horizontally in rotated frame
\begin{scope}[rotate=\planeAngle]

  % draw plane (a long rectangle strip)
  \fill[gray!20] (-0.5,-0.4) -- (5,-0.4) -- (5,0.4) -- (-0.5,0.4) -- cycle;
  \draw[black,line width=0.6pt] (-1,0) -- (6,0);

  % place square center at x = 2.2 along rotated axis
  \coordinate (C) at (2.2, \s + \gap);
  \coordinate (C1) at (3.5, \s + \gap);
  \coordinate (C2) at (1.1, \s + \gap);

  % draw square (centered at C)
  \draw[fill=blue!5] ($(C)+(-\s,-\s)$) rectangle ++(2*\s,2*\s);
  \draw ($(C)+(0,0)$) node {$m$};

  \draw[fill=blue!50,  opacity=0.2, dashed] 
  ($(C1)+(-\s,-\s)$) rectangle ++(2*\s,2*\s);
  % \draw[fill=blue!50, fill opacity=0.1] 
  % ($(C2)+(-\s,-\s)$) rectangle ++(2*\s,2*\s);

  % draw velocity vector (down the plane)
  %\draw[->, line width=1pt] ($(C)+(0,-\s)$) -- ++(0,-\arrowlen*0.55) node[midway,right] {$\mathbf{v}$};

  % gravity at center: draw downward in world frame (rotate back)
\end{scope}

% Transform C back to world coords by rotating
\coordinate (Cw) at ($(C)$);
\coordinate (Cworld) at ($(Cw)$);
% Actually the above C was defined in rotated scope; easier: recompute Cworld
\coordinate (Cworld) at ($( {2.2*cos(\planeAngle) - ( \s+\gap)*sin(\planeAngle) } , {2.2*sin(\planeAngle) + ( \s+\gap)*cos(\planeAngle) } )$);

% draw gravity (mg) downward (world vertical)
%\draw[->, line width=1pt] (Cworld) -- ++(0,-\arrowlen) node[below] {$m\mathbf{g}$};

% draw normal N (perpendicular to plane) at contact point
% contact point on plane under the cube (project center to plane)
\coordinate (proj) at ($(Cworld) + ({( \s+\gap)*sin(\planeAngle)},{-(\s+\gap)*cos(\planeAngle)})$);
\coordinate (applicationPoint) at (2.2 - 2*\s, \s + \gap);
% Normal direction (unit vector perpendicular to plane, pointing away from plane)
%\draw[->, line width=1pt] (proj) -- ++({\arrowlen*cos(\planeAngle)},{\arrowlen*sin(\planeAngle)}) node[right] {$\mathbf{N}$};

% Friction direction (parallel to plane, up the plane if block slides down)
\draw[->, line width=1pt] (applicationPoint) -- ++({ \arrowlen*cos(\planeAngle)},{ \arrowlen*sin(\planeAngle)}) node[above left] {${u}$};

% Draw components of gravity relative to plane: g_parallel and g_perp (dashed)
% Vector from center along negative plane-parallel direction
%\draw[dashed] (Cworld) -- ++({\arrowlen*sin(\planeAngle)},{-\arrowlen*cos(\planeAngle)}) coordinate (gpar);
%\draw[->, dashed] (Cworld) -- ++({\arrowlen*cos(90-\planeAngle)*0.9*sin(1)},{- \arrowlen}); % no-op to show dashed style (kept for clarity)

% Instead compute exact components using trig:
% g_parallel: along plane downward (points down the slope)
%\draw[->, blue!60!black, line width=0.9pt] (Cworld) -- ++({\arrowlen*sin(\planeAngle)},{- \arrowlen*cos(\planeAngle)}) node[midway, below right] {$mg\sin\theta$};
% g_perp: perpendicular to plane (points into plane)
%\draw[->, red!70!black, line width=0.9pt] (Cworld) -- ++({-\arrowlen*cos(\planeAngle)},{- \arrowlen*sin(\planeAngle)}) node[midway, left] {$mg\cos\theta$};

% draw angle label theta at origin between horizontal and plane
%\draw[->] (0.8,0.18) arc (0:\planeAngle:0.8) node[right, yshift=2] {$\theta$};

% Axes for reference
\draw[->] (-0.5,0) -- (6,0) node[above] {$y$};
\draw[-, line width=2pt] (-0.5,-0.4) -- (5,-0.4) node[below] {$y_{\max}$};
\draw[-, line width=1pt] (5,-0.4) -- (-0.5,-0.4) node[below] {$y_{\min}$};
%\draw[->] (0,-1.2) -- (0,3.2) node[above] {$y$};
\draw[->] (3.0,0.3) -- (4.3,0.3) node[pos=1.2] {$\dot{y}$};

% small note
%\node[align=left, right] at (6.2,2.6) {Square mass sliding \\ on an inclined plane};

\end{tikzpicture}
\end{minipage}
%\hfill
%\begin{minipage}{0.55\columnwidth}
%How to account for the limit on $y$?
%	
%\end{minipage}

The goal is to steer the mass to a desired position $y_{\rm d}$ while respecting 
\begin{itemize}
	\item $u_{\min}\leq u\leq u_{\max}$, 
	\item $y_{\min}\leq y\leq y_{\max}$ and $\dot{y}_{\min}\leq \dot{y}\leq \dot{y}_{\max}$
\end{itemize}


\end{frame}

\begin{frame}{Handling Inequality Constraints - Velocity Constraint}
	\hypertarget{velocity constraint}{}
	Same reasoning\footnote[frame]{Same steps apply for $h=\dot{y} -\dot{y}_{\min}$.}\hyperlink{CBF applications}{\beamerbutton{Back to CBF}}
	\begin{minipage}{0.4\columnwidth}
	\begin{align*}
		h&=\dot{y}_{\max} -\dot{y}\\
		\dot{h} &= -\ddot{y}
	\end{align*}
	\end{minipage}
	%\hfill
	%\begin{minipage}{0.55\columnwidth}
		
		Assume there exists $\phi_3(h)$ such that if $\dot{h} = \phi_3(h)$ then $h(t)=\alpha e^{-\lambda t}$. 
		% Note that if $h(t_0)\geq0 \ \Rightarrow h(t)\geq0, \  \forall t\geq t_0 \ \Rightarrow {\color{red}y(t)\leq y_{\max}}$
	%\end{minipage}
		Then, if 
	\begin{equation*}
	\dot{h} \geq \phi_3(h)~\left(\Leftrightarrow\ddot{y}\leq-\phi_3(h)\right), \ \text{and~} h(t_0)\geq0~\left(\Leftrightarrow\dot{y}(t_0)\leq \dot{y}_{\max}\right),
	\end{equation*}
	thus by virtue of the Comparison Lemma
	\begin{equation}
		h(t)\geq\alpha e^{-\lambda t}\geq0, \  \forall t\geq t_0 \ \Rightarrow {\color{red}\dot{y}(t)\leq \dot{y}_{\max}}
	\end{equation}
\end{frame}

\begin{frame}{Handling Inequality Constraints - Velocity Constraint}
QP is formulated as 
\begin{align*}
	\begin{split}
	 		\underset{\decisionVar=\begin{bmatrix}u,\ddot{y}\end{bmatrix}}{\min}&\frac{1}{2}
	 		\begin{Vmatrix}
	 			u-\phi_1(\vectorX)
	 		\end{Vmatrix}^2\\
		\rm{S.t:~}&u_{\min}\leq u\leq u_{\max}\\
							&\ddot{y}\leq-\phi_2(\vectorX_h)\\
							&\ddot{y}\leq-\phi_3(h)\\
							&m\ddot{y}=u
	 	\end{split}
\end{align*}
\end{frame}

\begin{frame}{Handling Inequality Constraints}
	\begin{itemize}
		\item In general, finding the feedback terms $\phi_1(\vectorX),\phi_2(\vectorX_h)$ and $\phi_3(h)$ is a relatively easier problem
		\item It relies on the control theory (Lyapunov theory, advanced control techniques, etc.), and control objectives (asymptotic/exponential/robust stability, etc.) e.g.:
		\begin{align*}
		\begin{split}
			\phi_1(\vectorX)&=-k_px - k_d\dot{x}\\
			\phi_2(\vectorX_h)&=-k_ph - k_d\dot{h}~\text{(with careful choice\footnotemark of $k_p$ and $k_d$)}\\
			\phi_3(h)&=-\lambda h
		\end{split}
		\end{align*}
		\addtocounter{footnote}{-3}
		\footnotetext{See~\cite{djeha2020ral}}
	\end{itemize}
\end{frame}
\begin{frame}{Handling Inequality Constraints}
	How does QP control relates to LQR approach?
	\begin{minipage}{0.45\columnwidth}
		\begin{align*}
			\begin{split}
				\underset{\decisionVar}{\min}&\frac{1}{2}\decisionVar^T\matriceH\decisionVar +\vectorH^T\decisionVar \\
				\rm{S.t:~}&\mathbf{C}_{\rm ineq}\decisionVar\leq\bm{d}_{\rm ineq}\\
				&\mathbf{C}_{\rm eq}\decisionVar=\bm{d}_{\rm eq}
			\end{split}
		\end{align*}
	\end{minipage}
	\begin{minipage}{0.45\columnwidth}
		\begin{align*}
			\underset{\vectorU}{\min}&\frac{1}{2}\int_{0}^{t}\left(\vectorX^T\matriceQ\vectorX+\vectorU^T\matriceR\vectorU\right)\\
			\rm{S.t:~}&\vectorXdot = \matriceA\vectorX+\matriceB\vectorU
		\end{align*}
	\end{minipage}
	\begin{itemize}
		\item Both have quadratic cost-function\alert<2>{... and that's it!}  
		\only<3->{\item They address different questions, and are solved differently}
	\end{itemize}
\end{frame}
\begin{frame}{Handling Inequality Constraints}
%	\begin{minipage}{0.45\columnwidth}
%		content...
%	\end{minipage}
\textbf{LQR:}
\begin{itemize}
	\item Problem solved analytically, offline
	\item Finds the optimal feedback gains that balance between  convergence speed and control effort 
	\item Explicit inequality constraints on the control or state are not handled
\end{itemize}
\textbf{QP:}
\begin{itemize}
	\item Problem solved numerically, online
	\item Finds the optimal control input based on the current state, cost-function and constraints
	\item Handle both equality and inequality constraints
\end{itemize}
\end{frame}
\begin{frame}%{Handling Inequality Constraints}
\begin{minipage}{0.2\columnwidth}
	{Example 1}
	\begin{align*}
	y_d &= 2.5~{\rm m} \\
	u_{\max} &= 2.5~{\rm N}\\
	y_{\max} &= 4~{\rm m}\\
	\dot{y}_{\max} &= 2~{\rm m/s}
	\end{align*}
\end{minipage}
\put(10,0){\begin{minipage}{0.75\columnwidth}
	\centering
	% \href{run:Figures/Atlas Gets a Grip _ Boston Dynamics.mp4}{
		% \includegraphics[height=\columnwidth]{Screenshot from 2025-06-11 12-55-26.png}} 
	\embedvideo*{\includegraphics[width=\textwidth,height=6.35cm]{Screenshot from 2025-10-09 19-21-48.png}}{Figures/QP_expl_1_adjusted.mp4}[autoplay=false,showGUI=true]%\\
	%\makebox{\hspace{1.5cm}\centering\small \emph{Atlas} humanoid robot assisting workers}
\end{minipage}}
\end{frame}

\begin{frame}%{Handling Inequality Constraints}
\begin{minipage}{0.2\columnwidth}
	Example 2 
	\begin{align*}
	y_d &= 5~{\rm m}\\
	u_{\max} &= 2.5~{\rm N}\\
	y_{\max} &= 4~{\rm m}\\
	\dot{y}_{\max} &= 2~{\rm m/s}
	\end{align*}
\end{minipage}
\put(10,0){\begin{minipage}{0.75\columnwidth}
		\centering
		% \href{run:Figures/Atlas Gets a Grip _ Boston Dynamics.mp4}{
			% \includegraphics[height=\columnwidth]{Screenshot from 2025-06-11 12-55-26.png}} 
		\embedvideo*{\includegraphics[width=\textwidth,height=6.35cm]{Screenshot from 2025-10-09 19-22-18.png}}{Figures/Example_target_5.0.mp4}[autoplay=false,showGUI=true]
	\end{minipage}}
\end{frame}
\begin{frame}{Nonlinear systems}
	\begin{itemize}
		\item QP Control is OK for linear systems... but how about nonlinear systems?
		\begin{equation*}
			\vectorXdot = f(\vectorX,\vectorU)
		\end{equation*}
		\only<2->{\item \textbf{Sufficient condition:} The nonlinear system is control affine 
		\begin{equation*}
			\vectorXdot = f(\vectorX) + g(\vectorX)\vectorU, \vectorX\inR^n, \vectorU\inR^m
		\end{equation*}}
		\only<3->{\item Fortunately, a large class of nonlinear systems are control affine\footnote[frame]{Robotic arms, humanoids, quadrupeds, drones, unicycle, etc.}!}
	\end{itemize}
	% \only<3->{\footnotetext}
\end{frame}
%\begin{frame}
%	\centering
%	\Large Task-space QP Control
%\end{frame}
\begin{frame}{Task-space QP Control}
%	\begin{itemize}
		 \textbf{Task-space:}  space in which the state of a certain point of interest $\vectorY(\vectorX)\inR^p$ is monitored\footnote[frame]{In control theory jargon, $\vectorY(\vectorX)$ is the system output.}, e.g., 
		\begin{itemize}
			\item Coordinate of a point on the robot (Cartesian-space~$\mathbb{R}^3$)
			\item Orientation of a frame attached to the robot (orientation-space ${\rm SO}(3)$)
			\item Coordinate of an object in the robot field of view (image-space $\mathbb{R}^2$)
		\end{itemize}  
		 \only<2->{\textbf{Control goal:} find $\vectorU$ such that $\vectorY(\vectorX)\longrightarrow\vectorY_d$ given that  $\vectorXdot = f(\vectorX) + g(\vectorX)\vectorU$}
		 
		 \only<3->{\textbf{Challenge:} the system dynamics and $\vectorY(\vectorX)$ are nonlinear }
%	\end{itemize}	
\end{frame}
\begin{frame}{Task-space QP Control}
	\textbf{Solution:} forget about the nonlinear dynamics, focus on the task-state 
	\begin{itemize}
		\item Pose $\vectorE=\vectorY-\vectorY_d\inR^p$
		\item Derive $\vectorE$ $\rho$ times until the control input $\vectorU$ appears
		\begin{equation*}
		\vectorE^{\rho}=	L_f^\rho\vectorE(\vectorX) + L_gL_f^{\rho-1}\vectorE(\vectorX)\vectorU = \vectorV, \ \vectorV\inR^p
		\end{equation*}
		\only<2->{\item $\vectorZ=\begin{bmatrix}
			\vectorE \\\vectorEdot\\\vdots\\\vectorE^{\rho-1}
		\end{bmatrix}\inR^{p\rho}\Rightarrow
		\vectorZdot = \begin{bmatrix}
			\mathbf{0}_{p(\rho-1)\times p} & \mathbf{I}_{p(\rho-1)}\\
			\mathbf{0}_{p\times p}&\mathbf{0}_{p\times p(\rho-1)}
		\end{bmatrix}\vectorZ + 
		\begin{bmatrix}
			\mathbf{0}_{p(\rho-1)\times p} \\\mathbf{I}_{p\times p}
		\end{bmatrix}\vectorV$ 
		
		\item  $\vectorV$ is the task-space control input}
	\end{itemize}	
\end{frame}
\begin{frame}{Task-space QP Control}
	\begin{itemize}
		\item Design $\vectorV=\phi(\vectorZ)$ such that 
		\begin{equation*}
			\vectorZ\longrightarrow0\Rightarrow \vectorE\longrightarrow0\Rightarrow \vectorY\longrightarrow\vectorY_d
		\end{equation*}
		\only<2->{\item The task dynamics is linear $\Rightarrow$ it is sufficient to choose $\phi(\vectorZ) = -\matriceK\vectorZ$, with $\matriceK\inR^{p\times p\rho}$ is the stabilizing gains matrix}
		\only<3->{\item Map the task-space control $\vectorV$ to the system control input $\vectorU$
		\begin{equation*}
			L_f^\rho\vectorE(\vectorX) + L_gL_f^{\rho-1}\vectorE(\vectorX)\vectorU = -\matriceK\vectorZ
		\end{equation*}}
		\only<4->{\item 	This mapping is \textbf{linear} w.r.t $\vectorU$ because the system dynamics is nonlinear \textbf{control-affine}!}
	\end{itemize}
\end{frame}
\begin{frame}{Task-space QP Control}
	\begin{itemize}
		\item How to compute $\vectorU$?
		\item \textbf{Analytically:} using the Jacobian pseudo-inverse
		\begin{equation*}
			\vectorU = \left[L_gL_f^{\rho-1}\vectorE(\vectorX)\right]^\dagger\left(-L_f^\rho\vectorE(\vectorX)-\matriceK\vectorZ\right)
		\end{equation*}
		\only<2->{Does not account for limits on $\vectorU$: $\vectorU_{\min}\leq\vectorU\leq\vectorU_{\max}$}
		\only<3->{\item \textbf{Numerically:}
		\begin{align*}
			\underset{\vectorU}{\min}&\frac{1}{2}\norm{L_gL_f^{\rho-1}\vectorE(\vectorX)\vectorU + L_f^\rho\vectorE(\vectorX) + \matriceK\vectorZ}^2\\
			{\rm S.t:~}&\vectorU_{\min}\leq\vectorU\leq\vectorU_{\max}
			\end{align*}}
		\only<4->{Is it the \textbf{unique} solution ?} 
	\end{itemize}
\end{frame}
\begin{frame}{Task-space QP Control}
	Another possible solution: 
	\begin{align*}
		\underset{\vectorU}{\min}&\frac{1}{2}\norm{\vectorU}^2\\
		{\rm S.t:~}&L_gL_f^{\rho-1}\vectorE(\vectorX)\vectorU =- L_f^\rho\vectorE(\vectorX) - \matriceK\vectorZ\\
		&\vectorU_{\min}\leq\vectorU\leq\vectorU_{\max}
		\end{align*}
		
		\only<2->{Does it produce the same solutions as the first QP form?}
\end{frame}
\begin{frame}{Task-space QP Control}
	\begin{minipage}{0.4\columnwidth}
		\begin{align*}
			\underset{\vectorU}{\min}&\frac{1}{2}\norm{L_gL_f^{\rho-1}\vectorE(\vectorX)\vectorU + L_f^\rho\vectorE(\vectorX) + \matriceK\vectorZ}^2\\
			{\rm S.t:~}&\color{red}\vectorU_{\min}\leq\vectorU\leq\vectorU_{\max}
		\end{align*}
		If $\min$ cost-function $\in$ feasibility domain, then it is the solution $\vectorU^\star$;\\
		Otherwise $\vectorU$ is saturated to its max values
	\end{minipage}
	\begin{minipage}{0.4\columnwidth}
		\begin{figure}
			\includegraphics[height=0.54\textheight]{QP1_sol.pdf}
		\end{figure}
	\end{minipage}
\end{frame}
\begin{frame}{Task-space QP Control}
\vspace*{-0.5cm}
	\begin{minipage}{0.4\columnwidth}
		\begin{align*}
			\underset{\vectorU}{\min}&\frac{1}{2}\norm{\vectorU}^2\\
			{\rm S.t:~}&{\color{blue}L_gL_f^{\rho-1}\vectorE(\vectorX)\vectorU =- L_f^\rho\vectorE(\vectorX) - \matriceK\vectorZ}\\
			&\color{red}\vectorU_{\min}\leq\vectorU\leq\vectorU_{\max}
		\end{align*}
		$(1)~\vectorU^\star$ the closest point on the hyper-plane to the origin\\
		$(2)~\vectorU^\star$ on the feasibility domain boundary\\
		$(3)~$ QP fails to find a feasible solution
	\end{minipage}
	\begin{minipage}{0.4\columnwidth}
	\vspace*{0.2cm}
		\begin{figure}
			\includegraphics[height=0.55\textheight]{QP2_sol.pdf}
		\end{figure}
	\end{minipage}
\end{frame}
\begin{frame}{Task-space QP Control}	
\vspace*{-0.5cm}
%	\begin{minipage}{0.3\columnwidth}	
	\begin{align}\label{eq:QP form1}
		\begin{split}
			\underset{\vectorU}{\min}&\frac{1}{2}\norm{L_gL_f^{\rho-1}\vectorE(\vectorX)\vectorU + L_f^\rho\vectorE(\vectorX) + \matriceK\vectorZ}^2\\
			{\rm S.t:~}&\color{red}\vectorU_{\min}\leq\vectorU\leq\vectorU_{\max}
		\end{split}
		\end{align}
%	\end{minipage}
%	\begin{minipage}{0.3\columnwidth}
		\begin{align}\label{eq:QP form2}
			\begin{split}
			\underset{\vectorU}{\min}&\frac{1}{2}\norm{\vectorU}^2\\
			{\rm S.t:~}&{\color{blue}L_gL_f^{\rho-1}\vectorE(\vectorX)\vectorU =- L_f^\rho\vectorE(\vectorX) - \matriceK\vectorZ}\\
			&\color{red}\vectorU_{\min}\leq\vectorU\leq\vectorU_{\max}
			\end{split}
		\end{align}
%	\end{minipage}
	
	The solutions of \cref{eq:QP form1,eq:QP form2} are \textbf{equivalent}, but \eqref{eq:QP form2} may run into \textbf{infeasibility}!!
\end{frame}

%Let's assume there exists a feedback $u=\phi_1(\vectorX)$ such that the origin of the closed-loop system $f(\vectorX,\phi_1(\vectorX))$ is asymptotically stable.
% \begin{itemize}$y_{\min}\leq y\leq y_{\max}$
% 	\item What is the context of QP use in control
% 	\item How QP can be used in control context? %A bit of history
% 	\item 
% \end{itemize}	
% \end{frame}


\section{Dynamic Modeling}%QP control strategies %New possibilities wit QP
%%\begin{frame}{Multi-objective control notion}
%	\begin{figure}
%		\centering
%		\begin{tikzpicture}[auto, node distance=2cm,>=latex']
%			\node [block] (CRA) at (-1, 12) {Complex robotic application};
%			\node [sum] (comp) at (4,11) {}; 
%			\node [block] (adjust) at (6,11) {\color{red}Adjustment};
%			
%			
%		\end{tikzpicture}
%	\end{figure}
%\end{frame}
\begin{frame}
	\centering
	\Large Multi-Objective Control
\end{frame}
\subsection{Multi-Objective Control}
\begin{frame}{Redundancy}
	\begin{itemize}
		\item For some systems, the Jacobian matrix $  L_gL_f^{\rho-1}\vectorE(\vectorX)\inR^{p\times m}$ is wide: $p<m$
		\item Example: humanoid robot 
		\begin{itemize}
			\item Number of actuators $m=\sim30$
			\item Task: control the hand position $p=3$
		\end{itemize} 
		\item In this case, \begin{equation*}
			L_gL_f^{\rho-1}\vectorE(\vectorX)\vectorU =- L_f^\rho\vectorE(\vectorX) - \matriceK\vectorZ
		\end{equation*} may accept multiple $ \vectorU$ solutions\footnote[frame]{Namely, $\ker(L_gL_f^{\rho-1}\vectorE(\vectorX))\neq\emptyset$}, but all of theme acheive the task!
		\item This type of systems (robots) are called \emph{‘redundant’}: more DoF than required to perform a task
	\end{itemize}
\end{frame}
\begin{frame}{Redundancy}
	\begin{itemize}	
		\item \textbf{Idea:} From the set of all possible solutions, choose --\emph{if possible}-- the one that achieves \emph{simultaneously} other tasks 
		\item E.g., robotic arm
		\begin{itemize}
			\item Task 01: bring the hand to a desired position
			\item Task 02: keep the elbow up  
		\end{itemize}
		\item Why ‘if possible’? Depending on the system state, redundancy can be lost $\Rightarrow$ impossible to perform \emph{two tasks simultaneously}, only one task can be performed, e.g., for a robotic arm
		\begin{itemize}
			\item Task 01: bring the hand to a desired position and orientation
			\item Task 02: keep the elbow up
		\end{itemize}
		\item When two (or more) objectives cannot be met simultaneously (conflictual), a hierarchy needs to be defined
	\end{itemize}
\end{frame}
\begin{frame}{Tasks Hierarchy}
	\begin{itemize}
		\item What is the purpose of defining a hierarchy between tasks?
		\item Hierarchy defines how the redundancy should be resolved such that low-priority control objectives do not (or at least minimally) disturb the high-priority ones
		\item Two tasks hierarchical schemes exist:
%	\begin{itemize}
%		\item Strict
%		\item Soft/weighted
%	\end{itemize}
\end{itemize}
\begin{minipage}{.48\columnwidth}
	\begin{figure}
		\centering
		\includegraphics[height=0.38\textheight]{strict.png}
	%	\includegraphics[height=0.4\textheight]{weighted.png}content...
	\caption{Strict hierarchy}
	\end{figure}
\end{minipage}
\begin{minipage}{.48\columnwidth}
	\begin{figure}
		\centering
	%	\includegraphics[height=0.4\textheight]{strict.png}
		\includegraphics[height=0.4\textheight]{weighted.png}
		\caption{Soft/weighted hierarchy}
	\end{figure}
\end{minipage}
\end{frame}
\begin{frame}{Tasks Hierarchy}
	\textbf{Strict hierarchy:}
	\begin{itemize}
		 \item Strict order of priority is defined (lexicographic order)
		 \begin{equation*}
		 	{\rm Task~01}\succ{\rm Task~02}\succ\cdots\succ{\rm Task~N}
		 \end{equation*}
		 \item \textbf{Principle:} 
		 \begin{enumerate}
		 	\item Perform ${\rm Task~01}$;
		 	\item If redundancy is still available $\rightarrow$ perform ${\rm Task~02}$ such that ${\rm Task~01}$ is not disturbed; else keep solution in (1)
		 	\item If redundancy is still available $\rightarrow$ perform ${\rm Task~03}$ such that ${\rm Task~01}$ and ${\rm Task~02}$ are not disturbed; else keep solution in (2)...
		 \end{enumerate}
		 \item \textbf{Methodology:} Solve cascade of QPs
		 \begin{itemize}
		 	\item \textbf{Pros:} Helps solving infeasibility
		 	\item \textbf{Cons:} Time consuming, not all tasks are performed, order subjectivity
		 \end{itemize}	  
	\end{itemize}
\end{frame}
\begin{frame}{Tasks Hierarchy}
	\textbf{Soft hierarchy:}
	\begin{itemize}
		\item Tasks are sorted by scalar positive weights 
		\begin{align}\label{eq:cost-func weighted QP}
			w_1\norm{\matriceA_1\vectorU+\vectorB_1}^2 &+ w_2\norm{\matriceA_2\vectorU+\vectorB_2}^2 + \cdots +
			w_N\norm{\matriceA_N\vectorU+\vectorB_N}^2,\\
			&w_i\geq0,\ i\in\left\{1,\cdots,N\right\}
		\end{align}
		\item \textbf{Principle:} Solve redundancy such that each task is performed \emph{at best} according to its weight (tasks competition)
		\item \textbf{Methodology:} Solve only one QP which cost-function is~\eqref{eq:cost-func weighted QP}
		\begin{itemize}
			\item \textbf{Pros:} Solved efficiently, all tasks are performed proportionally to their weights
			\item \textbf{Cons:} Precision not guaranteed, potential infeasibility due to constraints conflict 
		\end{itemize}
	\end{itemize}
\end{frame}
\begin{frame}{Tasks Hierarchy}
	\begin{tcolorbox}[colback=red!5!white, colframe=red!50!black, title=Remark]
		In both strict and soft hierarchical schemes: 
		\begin{itemize}
			\item the constraints have  \textbf{always} a higher priority than the tasks within the QP structure
			\item The constraints have the same level of priority
		\end{itemize}
	\end{tcolorbox}
	
\end{frame}
%\begin{frame}{Multi-objective control - Example}
%	Example: Moving the base of a quadruped robot
%	\begin{minipage}{0.51\columnwidth}
%		\begin{figure}
%			\includegraphics[width=\columnwidth]{multi-task example.pdf}
%		\end{figure}
%	\end{minipage}
%	\hfill
%	\begin{minipage}{0.48\columnwidth}
%		\begin{itemize}
%			\item \textbf{Task:} move the base in the manipulability space
%			\item \textbf{Constraints:} 
%			\begin{itemize}
%				\item Keep end-effector fixed
%				\item Keep non-slipping contacts
%				\item Avoid self-collision
%				\item Respect joint position/velocity/torque constraints
%				\item Keep CoM projection inside the static equilibrium region
%			\end{itemize}
%		\end{itemize}
%		
%	\end{minipage}
%\end{frame}
%\subsection{Control Lyapunov Functions \& Control Barrier Functions}
\begin{frame}
	\centering
	\Large Control Lyapunov Function
\end{frame}
\begin{frame}{Control Lyapunov Function}
\textbf{Lyapunov stability theorem:}\\
Given a system
\begin{equation*}
	\vectorXdot = f(\vectorX),\ \vectorX\in{\cal D}\subset\mathbb{R}^n
\end{equation*} 
having the origin as an equilibrium point. 
\begin{theorem}
	If their exists a positive definite function (PDF) $V(\vectorX)$ which time derivative $\dot{V}(\vectorX)$ is negative definite in~${\cal D}$, then the system is asymptotically stable.
\end{theorem}
 
%\vspace{0.3cm}
\only<2->{ \textbf{How to use Lyapunov stability theorem for stabilization?}}  
\end{frame}

\begin{frame}{Control Lyapunov Function}
	Given a system
	\begin{equation*}
		\vectorXdot = f(\vectorX) + g(\vectorX)\vectorU
	\end{equation*} 
$\vectorU=\phi(\vectorX)$ is asymptotically stabilizing control-law if their exists a PDF $V(\vectorX)$ such that 
\begin{equation*}
	\dot{V}(\vectorX) =%\frac{\partial  V(\vectorX)}{\partial\vectorX}f(\vectorX,\phi(\vectorX))<0
	L_fV(\vectorX) + L_gV(\vectorX)\phi(\vectorX)<0
\end{equation*}

\only<2->{\begin{tcolorbox}[colback=white, colframe=gray!50]
		This is the basic principle for the design of several control techniques, e.g., Backstepping.
	\end{tcolorbox}}
\end{frame}
\begin{frame}{Control Lyapunov Function}
	Alternatively, we can just find \textbf{any} $\vectorU$ that satisfies\footnote[frame]{This concept is not new. It has been already introduced by Zvi Artntein in 1983 (See~\cite{artstein1983nonlinearAnalysis})}
	\begin{equation}\label{eq:CLF constraint}
		L_fV(\vectorX) + L_gV(\vectorX)\vectorU\leq-\alpha(\norm{\vectorX}),
	\end{equation}
	with $\alpha(\norm{\vectorX})$ strictly monotonically increasing function. 
	
	
	\vspace{0.3cm}\only<2->{
	In this case, $V$ is called a \textbf{Control Lyapunov Function (CLF)}.}
\end{frame}
\begin{frame}{Control Lyapunov Function}
	\begin{tcolorbox}[colback=green!5!white, colframe=green!50!black, title=CLF-QP advantages]
		\begin{itemize}
			\item CLF constraint~\eqref{eq:CLF constraint} \textbf{linear} w.r.t $\vectorU$ $\Rightarrow$ can be \textbf{enforced} by QP
			\item $\vectorU$ computed \textbf{numerically} $\Rightarrow$ No analytic formula $\vectorU=\phi(\vectorX)$ is required
			\item Non-uniqueness of $\vectorU$ : CLF constraint~\eqref{eq:CLF constraint} constructs a \textbf{whole set ${\Upsilon}_{\rm CLF}$ of asymptotically stabilizing control-laws} 
			\begin{equation*}
				{\Upsilon}_{\rm CLF}=\left\{\vectorU\inR^m:\eqref{eq:CLF constraint}\right\}
			\end{equation*}
		\end{itemize}
	\end{tcolorbox}	
	%No longer required to compute $\vectorU$ analytically, just enforce the CLF constraint within QP
	%\only<2->{\textbf{Imagine how this is such a powerful tool to enforce other types of stability!}}	
\end{frame}
\begin{frame}{Control Lyapunov Function}
	\vspace*{-0.5cm}
	\begin{minipage}{0.5\columnwidth}
		\begin{align*}
			\underset{\vectorU}{\min}&\frac{1}{2}\norm{\vectorU}^2\\
			{\rm S.t:~}&{\color{blue}L_fV(\vectorX) + L_gV(\vectorX)\vectorU\leq-\alpha(\norm{\vectorX})}\\
			&\color{red}\vectorU_{\min}\leq\vectorU\leq\vectorU_{\max}
		\end{align*}
		\begin{itemize}
			\item Feasibility domain $\subset\Upsilon_{\rm CLF}$
			\item Optimal solution $\vectorU^{\star}$: minimum norm among all the stabilizing controllers in  $\Upsilon_{\rm CLF}$
		\end{itemize}
		
	\end{minipage}
	\begin{minipage}{0.4\columnwidth}
		\vspace*{0.2cm}
		\begin{figure}
			\includegraphics[height=0.55\textheight]{CLF-QP_sol.pdf}
		\end{figure}
	\end{minipage}
\end{frame}
\begin{frame}{Control Lyapunov Function}
	Several types of CLF can be formalized depending on the kind of stability being sought:
	\begin{itemize}
		\item Exponential CLF
		\item Robust CLF
		\item Adaptive CLF
		\item etc.
	\end{itemize}
	Still ongoing research works on this topic.
\end{frame}
\begin{frame}
	\centering
	\Large Control Barrier Function
\end{frame}
\begin{frame}{Control Barrier Function}
		Given a system
	\begin{equation*}
		\vectorXdot = f(\vectorX) + g(\vectorX)\vectorU
	\end{equation*}
	Sometimes, the goal is to find $\vectorU$ such that $\vectorX$ \emph{\textbf{remains within a given set ${\cal C}$ forward in time}}... \only<2->{Need for a theoretical framework!}
	
	\only<3->{ 
	\begin{tcolorbox}[colback=gray!5!white, colframe=gray!50!black, title=Set invariance]
		A set $\setC$ is said to be \emph{forward invariant} if 
		\begin{align*}
			\forall\vectorX(t_0)\in{\cal C}\Rightarrow\vectorX(t)\in{\cal C}, \ \forall t\geq t_0,%\\
			%	&{\text{Otherwise}}~\vectorX(t)\longrightarrow{\cal C}
		\end{align*}
	\end{tcolorbox}}
%In addition, if
%	\begin{equation*}
%		\textit{In addition, if}~\vectorX(t_0)\notin{\cal C}
%	\end{equation*}	
\end{frame}
\begin{frame}{Control Barrier Function}
	\begin{figure}
		\centering
		\includegraphics[width=0.5\columnwidth]{safety_set.pdf}
		\caption{Forward invariance of the set $\setC$. \textbf{The origin does not necessarily belong to $\setC$!}}
	\end{figure}
\end{frame}
\begin{frame}{Control Barrier Function}
	The ${\cal C}$  can be defined using a \textbf{barrier function}\footnote[frame]{$h(\vectorX)$ can be seen as a ‘distance to the boundary of $\setC$’.} $h(\vectorX)$
	\begin{equation*}
		\setC = \left\{\vectorX\inR^n:h(\vectorX)\geq0\right\}
	\end{equation*}
	with $h(\vectorX)=0$ means that $\vectorX$ is at the boundary of $\setC$. \\
	
	Examples:
	\begin{itemize}
		\item $h(\vectorX) = \vectorX_{\max}-\vectorX$: keep the state below the upper bound
		\item $h(\vectorX)={\rm dist}({\rm CoM}(\vectorX), \setS)$: keep CoM inside equilibrium polygon $\setS$
		\item $h(\vectorX)={\rm dist}\left(\vectorY_{{\rm ee}_1}(\vectorX)-\vectorY_{{\rm ee}_2}(\vectorX)\right)-\delta_{\min}$: keep a minimum distance $\delta_{\min}$ between the bodies $1$ and $2 $
	\end{itemize} 
%The barrier function is general tool to deal with state-dependant constraints.}
\end{frame}
\begin{frame}{Control Barrier Function}
		\begin{tcolorbox}[colback=blue!5!white, colframe=blue!50!black, title=Fundamental result]
		If the barrier function $h(\vectorX)\geq0$ over the system trajectories than the set $\setC$ is forward invariant.
	\end{tcolorbox}
	%\vspace{0.5 cm}
	\only<2->{
	\begin{tcolorbox}[colback=green!5!white, colframe=green!50!black, title=Sufficient condition -- Lyapunov-like condition]
		If
		\begin{equation*}
			\dot{h}\geq-\alpha(h),
		\end{equation*}
		 with $\alpha(.)$ is a strictly monotonically increasing function, then $\setC$ is \textbf{forward invariant} and \textbf{asymptotically stable}.
	\end{tcolorbox}}
	
\end{frame}
\begin{frame}{Control Barrier Function}
		\begin{figure}
		\centering
		\includegraphics[height=0.7\textheight]{safety_set_asymptotic_stability.pdf}
		\caption{Forward invariance (green) and asymptotic stability (blue) of the set $\setC$.}
	\end{figure} 
\end{frame}
\begin{frame}{Control Barrier Function}
	The forward invariance and asymptotic stability of $\setC$ can be enforced by 
finding $\vectorU$ such that
\begin{equation}\label{eq:CBF constraint}
	\dot{h}=L_fh(\vectorX)+L_gh(\vectorX)\vectorU\geq-\alpha(h)
\end{equation}
\vspace{0.3cm}\only<2->{	
In this case, $h$ is called a \textbf{Control Barrier Function}.}

\textbf{General framework:} large class of state-dependent constraints can be considered
\begin{itemize}
	\item Position and velocity constraints (already seen!) 
	\hyperlink{position constraint}{\beamerbutton{position constraint}}
	\hyperlink{velocity constraint}{\beamerbutton{velocity constraint}}
	\item Task-space constrained manipulation
	\item collision avoidance: Self-collision, Obstacles, Multi-robot, etc.
	%	\item Singularity avoidance
\end{itemize}
\end{frame}
\begin{frame}{Control Barrier Function}
	\hypertarget{CBF applications}{}
\begin{tcolorbox}[colback=green!5!white, colframe=green!50!black, title=CBF-QP advantages]
	\begin{itemize}
		\item CBF constraint~\eqref{eq:CBF constraint} linear w.r.t $\vectorU$ $\Rightarrow$ can be enforced by QP
		\item CBF ensures the \textbf{system  safety}: keep the state in the safety region (strong theoretical background)
		\item Non-uniqueness of $\vectorU$ : CBF constraint~\eqref{eq:CBF constraint} constructs a \textbf{whole set ${\Upsilon}_{\rm CBF}$ of safe control-laws} 
		\begin{equation*}
			{\Upsilon}_{\rm CBF}=\left\{\vectorU\inR^m:\eqref{eq:CBF constraint}\right\}
		\end{equation*}
	\end{itemize}
\end{tcolorbox}	
\end{frame}
\begin{frame}{CLF-CBF-QP}
	Stability and safety can be ensured by combining CLF and CBF through QP\footnote[frame]{For the fundamentals see~\cite{ames2017tac,ames2019ecc}, for a recent survey, see~\cite{li2023jas}.}
	\begin{align*}
		\begin{split}
			\underset{\vectorU}{\min}&\frac{1}{2}\norm{\vectorU}^2\\
			{\rm S.t:~}&L_fV(\vectorX) + L_gV(\vectorX)\vectorU<-\alpha(\norm{\vectorX})~\text{(CLF constraint)}\\
			&L_fh(\vectorX)+L_gh(\vectorX)\vectorU\geq-\alpha(h)~\text{(CBF constraint)}
		\end{split}
	\end{align*}	
\end{frame}
%\begin{frame}{CLF-CBF-QP}
%	\begin{figure}
%		\centering
%		\begin{tikzpicture}[auto, node distance=2cm,>=latex']
%			% Nodes
%			\node [block] (CRA) at (-1, 12) {CLF};
%			\node [sum] (comp) at (4,11) {}; 
%			\node [block] (adjust) at (6,11) {\color{red}Adjustment};
%			\node [block] (Sim) at (-1,10.5) {Simulation};
%			\node (COP) at (2.2, 9) {Control objective primitives};
%		\end{tikzpicture}
%	\end{figure}
%\end{frame}
%\begin{frame}{CLF-CBF-QP}
%		\begin{tcolorbox}[colback=red!5!white, colframe=red!50!black, title=Issue]
%			\begin{itemize}
%				\item CLF and CBF constraints have the same level of priority
%				\item If the constraints are \textbf{in conflict }(cannot be satisfied simultaneously), QP will fail to find a solution
%				\only<2->{\item Need to prioritize: \textbf{‘Stability first, than safety’} or \textbf{‘Safety first, than stability’}}
%				\only<3->{\item \textbf{Reasonable choice:} relax stability, prioritize safety!}
%			\end{itemize}
%		\end{tcolorbox}	
%\end{frame}
%\begin{frame}{CLF-CBF-QP}
%	Relaxed CLF-CBF-QP through slack variable ${\color{red}\delta}$
%	\begin{align*}
%		\begin{split}
%			\underset{\vectorU,{\color{red}\delta}}{\min}&\frac{1}{2}\norm{\vectorU}^2+\frac{1}{2}{{\color{red}\delta}}^2\\
%			{\rm S.t:~}&L_fV(\vectorX) + L_gV(\vectorX)\vectorU\geq-\alpha(\norm{\vectorX})+{\color{red}\delta}~\text{(Relaxed CLF constraint)}\\
%			&L_fh(\vectorX)+L_gh(\vectorX)\vectorU\geq-\alpha(h)~\text{(CBF constraint)}
%		\end{split}
%	\end{align*}
%	\begin{itemize}
%		\item $\delta$ is minimized to keep it bounded and to enforce $\delta=0$ when no relaxation is needed
%		\item Relaxed CLF ($\equiv\dot{V}\leq\delta$): allows  the task error to grow \textbf{to ensure} safety 		
%		\item Stability study if $\delta\neq0$: use \textbf{Input-to-State-Stability} theory 
%	\end{itemize}	
%\end{frame}
%\begin{frame}{Multi-CLF-CBF-QP}
%	The multi-objective case is written as  
%	\begin{align*}
%		\begin{split}
%			\underset{\vectorU,{\boldsymbol{\delta}}}{\min}&\frac{1}{2}\norm{\vectorU}^2+\frac{1}{2}{\norm{\boldsymbol{\delta}}}^2\\
%			{\rm S.t:~}&L_fV_1(\vectorX) + L_gV_1(\vectorX)\vectorU\geq-\alpha_1(\norm{\vectorX})+{\delta_1}\\
%			&\vdots\\
%			&L_fV_k(\vectorX) + L_gV_k(\vectorX)\vectorU\geq-\alpha_k(\norm{\vectorX})+{\delta_k}\\
%			&L_fh_1(\vectorX)+L_gh_1(\vectorX)\vectorU\geq-\alpha_1(h_1)\\
%			&\vdots\\
%			&L_fh_l(\vectorX)+L_gh_l(\vectorX)\vectorU\geq-\alpha_l(h_l)
%		\end{split}
%	\end{align*}	
%\end{frame}
\begin{frame}{CLF-CBF-QP - Example}
A mass moving on a plane.
\begin{figure}
	\centering
	\includegraphics[width=0.45\columnwidth]{CLF-CBF_obs_avoidance_1.pdf}
\end{figure}
\end{frame}
\begin{frame}{CLF-CBF-QP - Example}
	A mass moving on a plane.
	Red path: CLF-QP.
	\begin{figure}
		\centering
		\includegraphics[width=0.45\columnwidth]{CLF-CBF_obs_avoidance_2.pdf}
	\end{figure}
\end{frame}
\begin{frame}{CLF-CBF-QP - Example}
	A mass moving on a plane.
	Blue path: CLF-CBF-QP. 
	\begin{figure}
		\centering
		\includegraphics[width=0.45\columnwidth]{CLF-CBF_obs_avoidance_4.pdf}
	\end{figure}
\end{frame}

\begin{frame}{CLF-CBF-QP}
	\begin{tcolorbox}[colback=green!5!white, colframe=green!50!black, title=CLF-CBF-QP advantages]
		\begin{itemize}
			\item The stability and safety problems are treated \textbf{separately}! 
			\item \textbf{Compactness:}  ensuring the  task error convergence, \textbf{while} guaranteeing the system safety by one single QP 
			\item CBF constraint acts a filter for the non-safe CLF control laws
			\item CBF significantly reduces the necessity of motion planning!
		\end{itemize}
	\end{tcolorbox}
\end{frame}
%\begin{frame}{CLF-CBF-QP}
%	Now, what if, for a given system, we want to find $\vectorU$ such that 
%	\begin{itemize}
%		\item $\vectorY$ converges toward $\vectorY_d$ ($\vectorE = \vectorY-\vectorY_d\longrightarrow0$), \emph{\textbf{while}}
%		\item ensuring the forward invariance a set $\setC$ defined for a given state-dependant constraint
%	\end{itemize}
%	\begin{tcolorbox}[colback=white, colframe=gray!50, title=Example]
%		We want a segway-like robot to keep its upright position while keeping the pitch angle between max and min bounds
%	\end{tcolorbox} 
%\end{frame}
%\begin{frame}{Control Barrier Function}
%	\textbf{CBF-QP applications:}
%	\begin{
%\end{frame}

%\begin{frame}{Control Barrier Function}
%	The set and the origin: the relation between them! What if the set $\setC$ shrinks until becoming a point (origin)!
%\end{frame}
%\input{Multi-robot Control.tex}
\section{Industrial Robots Control}
%\subsection{Multi-robot Control}
\begin{frame}
	\centering
	\Large Multi-Robot Control
\end{frame}
\begin{frame}{Multi-robot Control}
\textbf{Goal:}	Control 2 robotic arms with a single controller
	\begin{align*}
		&\matriceM_1\vectorQddot_1+\vectorC_1(\vectorQ_1,\vectorQdot_1)=\vectorTau_1\\
		&\matriceM_2\vectorQddot_2+\vectorC_2(\vectorQ_2,\vectorQdot_2)=\vectorTau_2\\
		&h = \delta(\vectorQ_1,\vectorQ_2) - \delta_{\min}\geq0
	\end{align*}
	\begin{figure}
		\centering
		\includegraphics[width=0.8\columnwidth]{02Pandas.pdf}
	\end{figure}
\end{frame}
\begin{frame}{Multi-robot Control}
	By posing,
	\begin{equation*}
		\vectorQ = \begin{bmatrix}
			\vectorQ_1\\\vectorQ_2
		\end{bmatrix},\
		\vectorTau = \begin{bmatrix}
			\vectorTau_1\\\vectorTau_2
		\end{bmatrix},\
		\matriceM = {\rm BlockDiag}(\matriceM_1, \matriceM_2),\
		\vectorC = {\rm BlockDiag}(\vectorC_1, \vectorC_2)
	\end{equation*}
	the unified dynamics is obtained
	\begin{equation*}
		\matriceM\vectorQddot+\vectorC(\vectorQ,\vectorQdot)=\vectorTau
	\end{equation*}
\end{frame}
\begin{frame}{Multi-robot Control}
	The two robots can be independantly controlled using a \textbf{centralized}   QP
	\begin{align*}
		\begin{split}
			\underset{\vectorTau,\vectorQddot}{\min}&\frac{1}{2}\norm{\vectorTau-\vectorTau_d}^2\\
			{\rm S.t:~}&\vectorTau_{\min}\leq \vectorTau\leq \vectorTau_{\max}\\
				&\vectorQddot_{\min}\leq \vectorQddot\leq \vectorQddot_{\max}\\
				&\matriceM\vectorQddot+\vectorC(\vectorQ,\vectorQdot)=\vectorTau
		\end{split}
	\end{align*}
\end{frame}
\begin{frame}{Multi-robot Control}
	\textbf{Goal:}	Control 2 robotic arms while  avoiding collision
	\begin{align*}
		&\matriceM\vectorQddot+\vectorC(\vectorQ,\vectorQdot)=\vectorTau\\
		&h = \delta(\vectorQ_1,\vectorQ_2) - \delta_{\min}\geq0
	\end{align*}
	\begin{figure}
		\centering
		\includegraphics[width=0.8\columnwidth]{02Pandas_collision.pdf}
	\end{figure}
\end{frame}
\begin{frame}{Multi-robot Control}
	In practice, convex shells encapsulating the robots' bodies are used for a continuous measuring of the distance
	\begin{align*}
		&\matriceM\vectorQddot+\vectorC(\vectorQ,\vectorQdot)=\vectorTau\\
		&h = \delta(\vectorQ_1,\vectorQ_2) - \delta_{\min}\geq0
	\end{align*}
	\begin{figure}
		\centering
		\includegraphics[width=0.8\columnwidth]{02Pandas_collision_convex_shell.pdf}
	\end{figure}
\end{frame}
\begin{frame}{Multi-robot Control}
	Same centralized QP can be formulated through CBF constraint
	\begin{align*}
		\begin{split}
			\underset{\vectorTau,\vectorQddot}{\min}&\frac{1}{2}\norm{\vectorTau-\vectorTau_d}^2\\
			{\rm S.t:~}&\vectorTau_{\min}\leq \vectorTau\leq \vectorTau_{\max}\\
			&\vectorQddot_{\min}\leq \vectorQddot\leq \vectorQddot_{\max}\\
			&\matriceM\vectorQddot+\vectorC(\vectorQ,\vectorQdot)=\vectorTau\\
			&L_fh+L_gh\vectorTau\geq-\alpha(h)
		\end{split}
	\end{align*}
\end{frame}
\begin{frame}{CLF-CBF-QP}
	\begin{picture}(0,0)
		%		\put(0,70){Robots are able to accomplish complex missions...}
		\put(10,-20){\begin{minipage}{0.95\columnwidth}
				\centering
				\embedvideo*{\includegraphics[width=\columnwidth]{Screenshot from 2025-11-17 22-54-50.png}}{Figures/HandOVer-MOntage.mp4}[autoplay=false,showGUI=true]\\
				\makebox{\hspace{0cm}\centering\small Safe multi-robot handover}
		\end{minipage}}
	\end{picture}
\end{frame}
\begin{frame}{CLF-CBF-QP}
	\begin{picture}(0,0)
		%		\put(0,70){Robots are able to accomplish complex missions...}
		\put(10,-20){\begin{minipage}{0.95\columnwidth}
				\centering
				\embedvideo*{\includegraphics[width=\columnwidth]{Screenshot from 2025-11-17 22-54-50.png}}{Figures/HandOver-TwoPandas-Planes.mp4}[autoplay=false,showGUI=true]\\
				\makebox{\hspace{0cm}\centering\small Safe multi-robot handover}
		\end{minipage}}
	\end{picture}
\end{frame}
%\begin{frame}{CLF-CBF-QP}
%	\begin{picture}(0,0)
%		%		\put(0,70){Robots are able to accomplish complex missions...}
%		\put(10,-20){\begin{minipage}{0.85\columnwidth}
%				\centering
%				\embedvideo*{\includegraphics[height=0.6\textheight]{Screenshot from 2025-11-17 22-54-50.png}}{Figures/HRP2 Handover.mp4}[autoplay=false,showGUI=true]\\
%				\makebox{\hspace{0cm}\centering\small Safe humanoid handover}
%		\end{minipage}}
%	\end{picture}
%\end{frame}
\begin{frame}{Multi-robot Control}
	Now, assume  02 robotic arms manipulating a rigid object
%	\begin{align*}
%		&\matriceM_1\vectorQddot_1+\vectorC_1(\vectorQ_1,\vectorQdot_1)=\vectorTau_1\\
%		&\matriceM_2\vectorQddot_2+\vectorC_2(\vectorQ_2,\vectorQdot_2)=\vectorTau_2
%	\end{align*}
	\begin{figure}
		\centering
		\includegraphics[width=0.8\columnwidth]{02Pandas01Object.pdf}
	\end{figure}
	\only<2->{
	\begin{itemize}
		\item The robotics arms can nolonger move freely!
		\item How to formulate the control problem?
	\end{itemize}}
\end{frame}
\begin{frame}{Multi-robot Control}
	\begin{figure}
		\centering
		\includegraphics[width=0.8\columnwidth]{02Pandas01Object_forces.pdf}
	\end{figure}
	\begin{itemize}
		\item Consider the object as a (unactuated) 6-DoF \textbf{‘robot’} to which contact forces $\vectorF_{i,j}$ are applied
		\item All the robots are coupled through the contact forces
		\item Contact forces work in pairs of action/reaction 
	\end{itemize}
\end{frame}
\begin{frame}{Multi-robot Control}
	\begin{align*}
	&\matriceM_1\vectorQddot_1+\vectorC_1(\vectorQ_1,\vectorQdot_1)=\vectorTau_1+\matriceJ_{1,3}^T\vectorF_{1,3}\\
	&\matriceM_2\vectorQddot_2+\vectorC_2(\vectorQ_2,\vectorQdot_2)=\vectorTau_2+\matriceJ_{2,3}^T\vectorF_{2,3}\\
	&\matriceM_3\vectorQddot_3+\vectorC_3(\vectorQ_3,\vectorQdot_3)=\matriceJ_{3,1}^T\vectorF_{3,1} + \matriceJ_{3,2}^T\vectorF_{3,2} 
\end{align*}
Thanks to Newton's third law: 
\begin{equation*}
	\vectorF_{1,3} = -\vectorF_{3,1}, \ \vectorF_{2,3} = -\vectorF_{3,2}
\end{equation*}
leading to 
	\begin{align*}
\matriceM_3\vectorQddot_3+\vectorC_3(\vectorQ_3,\vectorQdot_3)=-\matriceJ_{3,1}^T\vectorF_{1,3} - \matriceJ_{3,2}^T\vectorF_{2,3} 
\end{align*}
\end{frame}
\begin{frame}{Multi-robot Control}
The unified dynamics becomes 
\begin{align*}
	&\matriceM\vectorQddot+\vectorC(\vectorQ,\vectorQdot)=\matriceS\vectorTau + \matriceJ^T\vectorF\\
		&\vectorQ = \begin{bmatrix}
		\vectorQ_1\\\vectorQ_2\\\vectorQ_3
	\end{bmatrix},\vectorTau = \begin{bmatrix}
		\vectorTau_1\\\vectorTau_2
	\end{bmatrix},\vectorF=\begin{bmatrix}
	\vectorF_{1,3} \\\vectorF_{2,3}
	\end{bmatrix},\vectorC = {\rm BlockDiag}(\vectorC_1, \vectorC_2,\vectorC_3),\\
	&\matriceM = {\rm BlockDiag}(\matriceM_1, \matriceM_2,\matriceM_3), \matriceS^T=\begin{bmatrix}
		\mathbf{I}&\bm{0}&\bm{0} \\
		\bm{0}&\mathbf{I}& \bm{0}
	\end{bmatrix},\\
	&\matriceJ = \begin{bmatrix}
		\matriceJ_{1,3}&\bm{0}&-\matriceJ_{3,1} \\
		\bm{0}&\matriceJ_{2,3}&-\matriceJ_{3,2}\\
	 \end{bmatrix}
\end{align*}
\end{frame}
\begin{frame}{Multi-robot Control}
	Multi-robot QP is then formulated as 
	\begin{align*}
		\begin{split}
			\underset{\vectorTau,\vectorQddot,{\color{red}\vectorF}}{\min}&\frac{1}{2}\norm{\vectorTau-\vectorTau_d}^2 + \frac{1}{2}\norm{\vectorF_{1,3}-\vectorF_{{1,3}_d}}^2 + \frac{1}{2}\norm{\vectorF_{2,3}-\vectorF_{{2,3}_d}}^2\\
			{\rm S.t:~}&\vectorTau_{\min}\leq \vectorTau\leq \vectorTau_{\max}\\
			&\vectorQddot_{\min}\leq \vectorQddot\leq \vectorQddot_{\max}\\
			&\matriceM\vectorQddot+\vectorC(\vectorQ,\vectorQdot)=\vectorTau\\
			&\vectorF_{1,3}\in{\cal F}_{1,3}~\text{(Friction cone)}\\
			&\vectorF_{2,3}\in{\cal F}_{2,3}~\text{(Friction cone)}
		\end{split}
	\end{align*}
	\textbf{Important result:} The \textbf{unactuated} object becomes \textbf{actuated} through the \textbf{contact forces!} 
\end{frame}
%\section{Applications}


\begin{frame}[allowframebreaks]{Références}
  \footnotesize
  \bibliographystyle{apalike}
  \bibliography{biblio.bib}
\end{frame}
% \bibliographystyle{apalike}	
% \bibliography{biblio.bib} 
\end{document}
