\begin{frame}{Rotation matrix }
	\begin{figure}
		\centering
		\includegraphics[width=0.5\columnwidth]{2frames.pdf}
	\end{figure}
	\begin{itemize}
		\item 	The orientation of the frame $\setR_B$ w.r.t the frame $\setR_I$ can be represented by the rotation matrix $\matriceR_B^I=\begin{bmatrix} \vectorX_B & \vectorY_B& \vectorZ_B\end{bmatrix}\in\rm{SO}(3)$
		\item $\vectorX_B, \vectorY_B,\vectorZ_B\inR^3$ are expressed in $\setR_I$
	\end{itemize}

\end{frame}
\begin{frame}{Rotation matrix }
	\begin{itemize}
		\item The columns of $\matriceR_B^I$ are orthogonal to each other: $\vectorX_B^T\vectorY_B=0$, $\vectorY_B^T\vectorZ_B=0$, $\vectorX_B^T\vectorZ_B=0$
		\item The columns of $\matriceR_B^I$ are normal: $\vectorX_B^T\vectorX_B=1$, $\vectorY_B^T\vectorY_B=1$, $\vectorZ_B^T\vectorZ_B=1$
		\item$\matriceR_I^B=\left(\matriceR_B^I\right)^{-1}=\left(\matriceR_B^I\right)^T$
		\item $\rm{SO}(3)$ denotes the group of $3\times3$ orthogonal matrices $\matriceR$ such that $\matriceR^T = \matriceR$ and $\det(\matriceR)=1$
		\item Sucessive rotations are expressed by rotation composition: $\matriceR_C^I = \matriceR_B^I\matriceR_C^B$. Meaning: perfirm a rotation $\matriceR_B^I$ w.r.t $\setR_I$, then perform a rotation $\matriceR_C^B$ w.r.t $\setR_B$
		\item \textbf{Remark:} rotation composition is generally not commutative!
	\end{itemize}
\end{frame}
\begin{frame}{Rotation matrix }
	\begin{itemize}
		\item If $\vectorV^B\inR^3$ is expressed in the frame $\setR_B$, then it can be expressed in the frame $\setR_I$ by $\vectorV=\matriceR_B^I\vectorV^B$
		\item Rotation matrix time derivative: 
		\begin{align*}
			\matriceRdot_B^I&=\matriceR_B^I\left[\vectorOmega_B^B\times\right]\\
			\matriceRdot_B^I&=\left[\vectorOmega_B^I\times\right]\matriceR_B^I
		\end{align*}
		with $\vectorOmega_B^B,\vectorOmega_B^I\inR^3$ are the body and world expressed angular velocity, respectively, and $\left[\star\times\right]\inR^{3\times 3}$ is the skew-matrix of the vector $\star$
		\item Orientation can also be represented by: Euler angles, Angle-axis, Quaternions
	\end{itemize}
\end{frame}
\begin{frame}{Rigid body kinematics}
	A rigid body kinematics is completely described by the \emph{position} and \emph{orientation} (\emph{pose}) of the frame $\setR_B$ (attached to the rigid body) w.r.t a reference frame $\setR_I$. 
	\begin{figure}
		\centering
		\includegraphics[width=0.6\columnwidth]{rigid-body.pdf}
	\end{figure}
\end{frame}
\begin{frame}{Rigid body kinematics}
The rigid body position is given by $\boldsymbol{o}_B\inR^3$ (expressed in $\setR_I$), whereas the orientation can be defined by the rotation matrix $\matriceR_B^I$%=\begin{bmatrix} \vectorX_B & \vectorY_B& \vectorZ_B\end{bmatrix}\in \rm{SO}(3)$
	\begin{figure}
		\centering
		\includegraphics[width=0.5\columnwidth]{rigid-body.pdf}
	\end{figure}
\end{frame}

\begin{frame}{Rigid body kinematics}
	The rigid body pose can be compactly represented by a homogeneous transformation matrix 
	\begin{equation*}
		\matriceT_B^I=\begin{bmatrix}
			\matriceR_B^I & \vectorO_B \\ \boldsymbol{0}_{1\times 3} &1
		\end{bmatrix}\in {\rm SE}(3)
	\end{equation*}
	
	\begin{figure}
		\centering
		\includegraphics[width=0.5\columnwidth]{rigid-body.pdf}
	\end{figure}
\end{frame}
\begin{frame}{Homegeneous transformation}
%	\textbf{Note:}
	\begin{itemize}
		\item The homogeneous transformation matrix performs a translation and a rotation of a rigid body without changing its dimensions while preserving the right handedness 
		\item Case of a pure translation:
		\begin{equation*}
			\matriceT_B^I=\begin{bmatrix}
				\matriceI_3 & \vectorO_B \\ \boldsymbol{0}_{1\times 3} &1
			\end{bmatrix}
		\end{equation*} 
		\item Case of a pure rotation:  
		\begin{equation*}
			\matriceT_B^I=\begin{bmatrix}
				\matriceR_B^I  & \boldsymbol{0}_{3\times 1}  \\ \boldsymbol{0}_{1\times 3} &1
			\end{bmatrix}
		\end{equation*}
	\end{itemize}
\end{frame}
\begin{frame}{Homegeneous transformation}
	The inverse of the homogeneous transformation matrix is given by 
	\begin{align*}
		\left(\matriceT_B^I\right)^{-1} = \matriceT^B_I &= \begin{bmatrix}
			\left(\matriceR_B^I\right)^T & -\left(\matriceR_B^I\right)^T \vectorO_B\\
			\vectorZero_{1\times3} &1
		\end{bmatrix} \\
		&=\begin{bmatrix}
			\matriceR^B_I & -\matriceR^B_I \vectorO_B\\
			\vectorZero_{1\times3} &1
		\end{bmatrix}
	\end{align*} 
\end{frame}
\begin{frame}{Homegeneous transformation}
	In homogeneous coordinates, a point is represented by 
	\begin{equation*}
		\underline{\vectorP} = \begin{bmatrix}
			x\\y\\z\\1
		\end{bmatrix}
	\end{equation*}
	while a velocity vector is represented as 
	\begin{equation*}
		\underline{\vectorV} = \begin{bmatrix}
			v_x\\v_y\\v_z\\0
		\end{bmatrix}
	\end{equation*}
\end{frame}
\begin{frame}{Homegeneous transformation}
	Given a point $P$ with homogeneous coordinates $\underline{\vectorP}_B$ expressed in the frame $\setR_B$. Its coordinates $\underline{\vectorP}$ expressed in $\setR_I$ can be computed as 
	\begin{align*}
		\underline{\vectorP} &= \matriceT_B^I\underline{\vectorP}_B 
	\end{align*}
	\begin{figure}
		\centering
		\includegraphics[width=0.5\columnwidth]{TransformationMatrix-PositionVector}
	\end{figure}
\end{frame}
\begin{frame}{Homegeneous transformation}
	Given a velocity vector with homogeneous coordinates $\underline{\vectorV}_B$ expressed in $\setR_B$. Its coordinates $\underline{\vectorV}$ expressed in $\setR_I$ can be computed as 
	\begin{align*}
		\underline{\vectorV} &= \matriceT_B^I\underline{\vectorV}_B 
	\end{align*}
	\begin{figure}
		\centering
		\includegraphics[width=0.55\columnwidth]{TransformationMatrix-VelocityVector}
	\end{figure}
\end{frame}
%\begin{frame}{Homegeneous transformation}
%	\begin{itemize}	
%		\item The position and orientation of the rigid body w.r.t the frame $\setR_I$ can be completely described by $\matriceT_B^I$: 
%		\begin{align*}
%			\begin{bmatrix}
%				\vectorO_B \\ 1
%			\end{bmatrix} &= \matriceT_B^I\begin{bmatrix}
%				0\\0\\0 \\ 1
%			\end{bmatrix}\\
%			\matriceR_B^I &= \matriceT_B^I(1:3,1:3)
%		\end{align*}
%	\end{itemize}
%\end{frame}
%\begin{frame}{Homegeneous transformation}
%	Consider a point $P$ on the rigid body. The position $\boldsymbol{p}_I$ of $P$ w.r.t the frame $\setR_I$ can be expressed  by the homogeneous transformation 
%	\begin{equation*}
%		{\color{blue}\vectorP_I} = {\color{red}\vectorO_B} + \matriceR_B^I{\color{SeaGreen}\boldsymbol{p}_B}
%	\end{equation*}
%	where $\boldsymbol{p}_B$ is the position of the point $P$ in the frame $\setR_B$
%		\begin{figure}
%		\centering
%		\includegraphics[width=0.5\columnwidth]{rigid-body_TransformationMatrix.pdf}
%	\end{figure}
%\end{frame}
%\begin{frame}{Homegeneous transformation}
%Alternatively, the position $\vectorP_I$ can be computed by the homegeneous transformation matrix $\matriceT_B^I$
%	\begin{equation*}
%		\begin{bmatrix}
%			\boldsymbol{p}_I \\ 1
%		\end{bmatrix} = \matriceT_B^I\begin{bmatrix}
%			\boldsymbol{p}_B \\ 1
%		\end{bmatrix}, 
%	\end{equation*}
%	\begin{figure}
%		\centering
%		\includegraphics[width=0.5\columnwidth]{rigid-body_TransformationMatrix.pdf}
%	\end{figure}
%\end{frame}
\begin{frame}{Homegeneous transformation}
The {\color{blue}{position}} and {\color{purple}orientation} of the  frame $\setR_C$ w.r.t to $\setR_I$ can be computed through the homogeneous transformation matrix composition
			
	\begin{minipage}{0.3\columnwidth}
		\begin{align*}
			\matriceT_C^I&=\matriceT_B^I\matriceT_C^B\\
			&=\begin{bmatrix}
				\matriceR_B^I&{\color{red}\vectorO_B}\\\vectorZero_{1\times3}&1
			\end{bmatrix}\begin{bmatrix}
				\matriceR_C^B&{\color{SeaGreen}\vectorO_C^B}\\\vectorZero_{1\times3}&1
			\end{bmatrix} \\
			&=\begin{bmatrix}
				\matriceR_B^I\matriceR_C^B&{\color{blue}\vectorO_B+\matriceR_B^I\vectorO_C^B}\\\vectorZero_{1\times3}&1
			\end{bmatrix}\\
			&=\begin{bmatrix}
				{\color{purple}\matriceR_C^I}&{\color{blue}\vectorO_B+\matriceR_B^I\vectorO_C^B}\\\vectorZero_{1\times3}&1
			\end{bmatrix}
		\end{align*}
	\end{minipage}
	\hfill
	\begin{minipage}{0.4\columnwidth}
		\begin{figure}
			\centering
			\includegraphics[width=\columnwidth]{rigid-body_TransformationMatrix_rot.pdf}
		\end{figure}
	\end{minipage}
\end{frame}
\begin{frame}{Forward kinematics}
In the case of articulated robot, the position and orientation of the end-effector frame  $\setR_{ee}$ w.r.t to $\setR_I$ can be obtained through the homogeneous transformation matrix composition as a function of the joint variables
	
	\begin{minipage}{0.35\columnwidth}
		\begin{align*}
			\matriceT_{ee}^I&=\matriceT_B^I\matriceT_C^B\matriceT_D^C\matriceT_E^D\matriceT_F^E\matriceT_{ee}^{F}\\
		\end{align*}
		$\matriceT_{ee}^I$ is denoted as the \textbf{forward kinematic model}
	\end{minipage}
	\hfill
	\begin{minipage}{0.6\columnwidth}
		\begin{figure}
			\centering
			\includegraphics[height=0.4\textheight]{rigid-tree_TransformationMatrix.pdf}
		\end{figure}
	\end{minipage}
\end{frame}
\begin{frame}{Forward kinematics}
\begin{itemize}
	\item The homogeneous transformation matrix $\matriceT_{ee}^I$ depends on the choice of the frames placement
	\item Need for a \textbf{standard method} to decribe the relative transformation matrices between each pair of bodies with a \textbf{minimum number of parameters}\footnote[frame]{A homogeneous transformation matrix needs 9 parameters.}
\end{itemize}
	
%	\begin{minipage}{0.2\columnwidth}
		\begin{align*}
			\matriceT_{ee}^I&=\begin{bmatrix}
				{\matriceR_{ee}^I}&{\vectorP_{ee}}\\\vectorZero_{1\times3}&1
			\end{bmatrix}
		\end{align*}
%	\end{minipage}
\end{frame}
\begin{frame}{Denavit-Hartenberg method}
DH convention is based on the following rules for the frames placement
\begin{itemize}
	\item Choose axis $z_j$ along the axis of joint $j+1$
	\item Choose axis $x_j$ such that $x_j\perp z_{j-1}$ and intersects with it
\end{itemize}
\begin{figure}
	\centering
	\includegraphics[height=0.55\textheight]{bodies&joints.pdf}
\end{figure}
\end{frame}
\begin{frame}{Denavit-Hartenberg method}
The homogeneous transformation matrix $\matriceT_{j}^{j-1}$ is defined using 4 DH parameters

	\begin{minipage}{0.5\columnwidth}
		\begin{itemize}
			\item $\theta_j$: angle between $x_{j-1}$ and $x_j$ around $z_{j-1}$
			\item $d_j$: distance between $x_{j-1}$ and $x_j$ along $z_{j-1}$
			\item $\alpha_j$: angle between $z_{j-1}$ and $z_j$ around $x_{j}$
			\item $a_j$: distance between $z_{j-1}$ and $z_j$ along $x_{j}$
		\end{itemize}
	\end{minipage}
	\hfill
	\begin{minipage}{0.45\columnwidth}
			\begin{figure}
			\centering
			\includegraphics[width=\columnwidth]{DH.pdf}
		\end{figure}
	\end{minipage}
%	$a_j$ and $\alpha_j$ are always constant and depends on the placement of the joint on the robot structure, whereas $\theta_j$ and $d_j$ depends on the type of the joint:
%	\begin{itemize}
%		\item If joint $j$ is revolute $ \Rightarrow\theta_j$ is variable, $d_j$ constant
%		\item If joint $j$ is prismatic $ \Rightarrow d_j$ is variable, $\theta_j$ constant
%	\end{itemize} 
	

\end{frame}
\begin{frame}{Denavit-Hartenberg method}
	\begin{itemize}
		\item $a_j$ and $\alpha_j$ are \textbf{always constant} and depend on the placement of the joint on the robot structure
		\item $\theta_j$ and $d_j$ are related to the joint position $q_j$ and depend on the joint type:
		\begin{itemize}
			\item If joint $j$ is \textbf{revolute} $ \Rightarrow q_j=\theta_j$ is variable, $d_j$ constant
			\item If joint $j$ is \textbf{prismatic} $ \Rightarrow q_j=d_j$ is variable, $\theta_j$ constant
		\end{itemize} 
		\item Joint position $q_j$ is denoted as $q_j=\overline{\sigma}_j\theta_j+\sigma_jd_j$ ($\sigma=0$ revolute, $\sigma_j=1$ prismatic) 
	\end{itemize} 	
\end{frame}
\begin{frame}{Denavit-Hartenberg method}
	The homogeneous transformation matrix $\matriceT_{j}^{j-1}$ is computed as 
	\begin{align*}
		\matriceT_{j}^{j-1} &= \matriceR_{z_{j-1},\theta_j}\mathbf{Trans}_{z{j-1},d_j}\mathbf{Trans}_{x_j,a_j}\matriceR_{x_j,\alpha_j}\\
		\matriceT_{j}^{j-1} &= \begin{bmatrix}
			\cos(\theta_j)&-\sin(\theta_j)\cos(\alpha_j) &\sin(\theta_j)\sin(\alpha_j) & a_j\cos(\theta_j)\\
			\sin(\theta_j)&\cos(\theta_j)\cos(\alpha_j) &-\cos(\theta_j)\sin(\alpha_j) & a_j\sin(\theta_j)\\
			0&\sin(\alpha_j)&\cos(\alpha_j)&d_j\\
			0&0&0&1
		\end{bmatrix}
	\end{align*}
	The \textbf{forward kinematics model} of the end-effector is obtained as a function of $\vectorQ$
	\begin{equation*}
		\matriceT_{ee}^{0}(\vectorQ) = \matriceT_{1}^{0}(q_1) \matriceT_{2}^{1}(q_2)\ldots\matriceT_{n}^{ee}(q_{n})
	\end{equation*}
\end{frame}
\begin{frame}{Denavit-Hartenberg method}
	Example: 3 DoF (RRR) 
		
	\begin{minipage}{.4\columnwidth}
		\begin{itemize}
			\item Place the frames at each joint
			\item Determine the parameters $\theta_j$, $d_j$, $\alpha_j$ and $a_j$
		\end{itemize}
	\end{minipage}
	\hfill
	\begin{minipage}{.5\columnwidth}
		\begin{figure}
			\centering
			\includegraphics[height=0.7\textheight]{ex-DH-woFrames}
		\end{figure}
	\end{minipage}
\end{frame}
\begin{frame}{Denavit-Hartenberg method}
	Example: 3 DoF (RRR) 
		
	\begin{minipage}{.4\columnwidth}
$1^{\text st}$ step: place $z$ axis at each joint, $z_0$ axis is for the fixed frame\\
\textbf{Remark:} the frame at the end-effector is set assuming a fixed joint 
	\end{minipage}
	\hfill
	\begin{minipage}{.5\columnwidth}
		\begin{figure}
			\centering
			\includegraphics[height=0.7\textheight]{ex-DH-Z-Axis}
		\end{figure}
	\end{minipage}
\end{frame}
\begin{frame}{Denavit-Hartenberg method}
	Example: 3 DoF (RRR) 
	
	\begin{minipage}{.4\columnwidth}
		$2^{\text nd}$ step: place $x$ axis at each such that $x_i\perp z_{i-1}$ and intersect with it
	%	\textbf{Remark:} the frame at the end-effector is set assuming a fixed joint 
	\end{minipage}
	\hfill
	\begin{minipage}{.5\columnwidth}
		\begin{figure}
			\centering
			\includegraphics[height=0.7\textheight]{ex-DH-ZX-Axis}
		\end{figure}
	\end{minipage}
\end{frame}
\begin{frame}{Denavit-Hartenberg method}
	Example: 3 DoF (RRR) 
	
	\begin{minipage}{.4\columnwidth}
		$3^{\text rd}$ step: determine the parameters
		\begin{table}
			\begin{tabular}{|c|c|c|c|c|c|c|}
				\hline
				$j$&$\sigma_j$&$\theta_j$&$d_j$&$\alpha_j$&$a_j$\\ \hline\hline
				$1$&$.$&$.$&$.$&$.$&$.$\\ \hline
				$2$&$.$&$.$&$.$&$.$&$.$\\ \hline
				$3$&$.$&$.$&$.$&$.$&$.$\\ \hline
			\end{tabular}
		\end{table}
	\end{minipage}
	\hfill
	\begin{minipage}{.5\columnwidth}
		\begin{figure}
			\centering
			\includegraphics[height=0.7\textheight]{ex-DH-ZX-Axis}
		\end{figure}
	\end{minipage}
\end{frame}
\begin{frame}{Denavit-Hartenberg method}
	Example: 3 DoF (RRR) 
	
	\begin{minipage}{.4\columnwidth}
		$3^{\text rd}$ step: determine the parameters
		\begin{table}
			\begin{tabular}{|c|c|c|c|c|c|c|}
				\hline
				$j$&$\sigma_j$&$\theta_j$&$d_j$&$\alpha_j$&$a_j$\\ \hline\hline
				$1$&$0$&$q_1$&$l_1$&$\pi/2$&$0$\\ \hline
				$2$&$0$&$q_2$&$0$&$0$&$l_2$\\ \hline
				$3$&$0$&$q_3$&$0$&$0$&$l_3$\\ \hline
			\end{tabular}
		\end{table}
	\end{minipage}
	\hfill
	\begin{minipage}{.5\columnwidth}
		\begin{figure}
			\centering
			\includegraphics[height=0.7\textheight]{ex-DH-ZX-Axis}
		\end{figure}
	\end{minipage}
\end{frame}
\begin{frame}{Denavit-Hartenberg method}
	Example: 3 DoF (RRP) 
	
	\begin{minipage}{.4\columnwidth}
	\begin{itemize}
		\item Place the frames at each joint
		\item Determine the parameters $\theta_j$, $d_j$, $\alpha_j$ and $a_j$
	\end{itemize}
	\end{minipage}
	\hfill
	\begin{minipage}{.5\columnwidth}
		\begin{figure}
			\centering
			\includegraphics[width=\columnwidth]{ex2-DH-woFrames}
		\end{figure}
	\end{minipage}
\end{frame}
\begin{frame}{Denavit-Hartenberg method}
	Example: 3 DoF (RRP) 
	
	\begin{minipage}{.4\columnwidth}
$1^{\text st}$ step: place $z$ axis at each joint, $z_0$ axis is for the fixed frame\\
\textbf{Remark:} the frame at the end-effector is set assuming a fixed joint 
	\end{minipage}
	\hfill
	\begin{minipage}{.5\columnwidth}
		\begin{figure}
			\centering
			\includegraphics[width=\columnwidth]{ex2-DH-Z-Axis}
		\end{figure}
	\end{minipage}
\end{frame}
\begin{frame}{Denavit-Hartenberg method}
	Example: 3 DoF (RRP) 
	
	\begin{minipage}{.4\columnwidth}
		$2^{\text nd}$ step: place $x$ axis at each such that $x_i\perp z_{i-1}$ and intersect with it\\
%		\textbf{Remark:} the frame at the end-effector is set assuming a fixed joint 
	\end{minipage}
	\hfill
	\begin{minipage}{.5\columnwidth}
		\begin{figure}
			\centering
			\includegraphics[width=\columnwidth]{ex2-DH-ZX-Axis}
		\end{figure}
	\end{minipage}
\end{frame}
\begin{frame}{Denavit-Hartenberg method}
	Example: 3 DoF (RRP) 
	
	\begin{minipage}{.4\columnwidth}
			$3^{\text rd}$ step: determine the parameters
		\begin{table}
			\begin{tabular}{|c|c|c|c|c|c|c|}
				\hline
				$j$&$\sigma_j$&$\theta_j$&$d_j$&$\alpha_j$&$a_j$\\ \hline\hline
				$1$&$.$&$.$&$.$&$.$&$.$\\ \hline
				$2$&$.$&$.$&$.$&$.$&$.$\\ \hline
				$3$&$.$&$.$&$.$&$.$&$.$\\ \hline
			\end{tabular}
		\end{table} 
	\end{minipage}
	\hfill
	\begin{minipage}{.5\columnwidth}
		\begin{figure}
			\centering
			\includegraphics[width=\columnwidth]{ex2-DH-ZX-Axis}
		\end{figure}
	\end{minipage}
\end{frame}
\begin{frame}{Denavit-Hartenberg method}
	Example: 3 DoF (RRP) 
	
	\begin{minipage}{.4\columnwidth}
		$3^{\text rd}$ step: determine the parameters
		\begin{table}
			\begin{tabular}{|c|c|c|c|c|c|c|}
				\hline
				$j$&$\sigma_j$&$\theta_j$&$d_j$&$\alpha_j$&$a_j$\\ \hline\hline
				$1$&$0$&$q_1$&$0$&$0$&$l_1$\\ \hline
				$2$&$0$&$q_2$&$0$&$0$&$l_2$\\ \hline
				$3$&$1$&$0$&$q_3$&$\pi$&$0$\\ \hline
			\end{tabular}
		\end{table} 
	\end{minipage}
	\hfill
	\begin{minipage}{.5\columnwidth}
		\begin{figure}
			\centering
			\includegraphics[width=\columnwidth]{ex2-DH-ZX-Axis}
		\end{figure}
	\end{minipage}
\end{frame}
\begin{frame}{Denavit-Hartenberg method}
	For each of the previous examples, $\matriceT_j^{j-1}$ is computed for $j=1,2,3$. Then, the forward kinematic model is obtained by computing 
	\begin{equation*}
		\matriceT_{ee}^{0}(\vectorQ) = \matriceT_{1}^{0}(q_1) \matriceT_{2}^{1}(q_2)\matriceT_{2}^{3}(q_3)
	\end{equation*}
%	\textbf{Remarks about DH method:}
%	\begin{itemize}
%		\item The body frame is located after the joint
%		\item For complex robot, this can be non-intuitive and prone to error
%	\end{itemize}
\end{frame}
\begin{frame}{Denavit-Hartenberg method}
	mDH convention is based on \emph{modified} rules for the frames placement
	\begin{itemize}
		\item Choose axis $z_j$ along the axis of joint $j$ (previously, joint $j+1$)
		\item Choose axis $x_{j-1}$ such that $x_{j-1}\perp z_{j}$ and intersects with it (previously, axis $x_j$)
	\end{itemize}
	\begin{figure}
		\centering
		\includegraphics[height=0.5\textheight]{bodies&joints.pdf}
	\end{figure}
\end{frame}
\begin{frame}{Modified Denavit-Hartenberg method}
	The homogeneous transformation matrix $\matriceT_{j}^{j-1}$ is defined using 4 mDH parameters
	
	\begin{minipage}{0.5\columnwidth}
		\begin{itemize}
			\item $a_j$: distance between $z_{j-1}$ and $z_j$ along $x_{j-1}$
			\item $\alpha_j$: angle between $z_{j-1}$ and $z_j$ around $x_{j-1}$			
			\item $d_j$: distance between $x_{j-1}$ and $x_j$ along $z_{j}$
			\item $\theta_j$: angle between $x_{j-1}$ and $x_j$ around $z_{j}$		
		\end{itemize}
	\end{minipage}
	\hfill
	\begin{minipage}{0.45\columnwidth}
		\begin{figure}
			\centering
			\includegraphics[width=\columnwidth]{mDH.pdf}
		\end{figure}
	\end{minipage}
	%	$a_j$ and $\alpha_j$ are always constant and depends on the placement of the joint on the robot structure, whereas $\theta_j$ and $d_j$ depends on the type of the joint:
	%	\begin{itemize}
		%		\item If joint $j$ is revolute $ \Rightarrow\theta_j$ is variable, $d_j$ constant
		%		\item If joint $j$ is prismatic $ \Rightarrow d_j$ is variable, $\theta_j$ constant
		%	\end{itemize} 	
\end{frame}
\begin{frame}{Denavit-Hartenberg method}
Another difference is the order of the homogeneous transformation matrix $\matriceT_{j}^{j-1}$ computation 
	\begin{align*}
		\matriceT_{j}^{j-1} &=\mathbf{Trans}_{x_{j-1},a_j}\matriceR_{x_{j-1},\alpha_j}\mathbf{Trans}_{z{j},d_j} \matriceR_{z_{j},\theta_j}\\
		\matriceT_{j}^{j-1} &= \begin{bmatrix}
			\cos(\theta_j)&-\sin(\theta_j) &0 & a_j\\
			\cos(\alpha_j)\sin(\theta_j)&\cos(\alpha_j)\cos(\theta_j) &-\sin(\alpha_j) & -d_j\sin(\alpha_j)\\
			\sin(\alpha_j)\sin(\theta_j)&\sin(\alpha_j)\cos(\theta_j)&\cos(\alpha_j)&d_j\cos(\alpha_j)\\
			0&0&0&1
		\end{bmatrix}
	\end{align*}
	The \textbf{forward kinematics model} of the end-effector is obtained as a function of $\vectorQ$
	\begin{equation*}
		\matriceT_{ee}^{0}(\vectorQ) = \matriceT_{1}^{0}(q_1) \matriceT_{2}^{1}(q_2)\ldots\matriceT_{n}^{ee}(q_{n})
	\end{equation*}
\end{frame}
\begin{frame}{Modified Denavit-Hartenberg method}
	Similarly to DH method:
	\begin{itemize}
		\item $a_j$ and $\alpha_j$ are \textbf{always constant} and depend on the placement of the joint on the robot structure 
		\item Joint position $q_j$ is denoted as $q_j=\overline{\sigma}_j\theta_j+\sigma_jd_j$ ($\sigma=0$ revolute, $\sigma_j=1$ prismatic) 
	\end{itemize} 	
\end{frame}
\begin{frame}{Modified Denavit-Hartenberg method}
	Example: 3 DoF (RRR) 
	
	\begin{minipage}{.4\columnwidth}
		\begin{itemize}
			\item Place the frames at each joint
			\item Determine the parameters $\theta_j$, $d_j$, $\alpha_j$ and $a_j$
		\end{itemize}
	\end{minipage}
	\hfill
	\begin{minipage}{.5\columnwidth}
		\begin{figure}
			\centering
			\includegraphics[height=0.7\textheight]{ex-DH-woFrames}
		\end{figure}
	\end{minipage}
\end{frame}
\begin{frame}{Modified Denavit-Hartenberg method}
	Example: 3 DoF (RRR) 
	
	\begin{minipage}{.4\columnwidth}
		$1^{\text st}$ step: place $z$ axis at each joint, $z_0$ axis is for the fixed frame\\
		\textbf{Remark:} the frame at the end-effector is set assuming a fixed joint 
	\end{minipage}
	\hfill
	\begin{minipage}{.5\columnwidth}
		\begin{figure}
			\centering
			\includegraphics[height=0.7\textheight]{ex-mDH-Z-Axis}
		\end{figure}
	\end{minipage}
\end{frame}
\begin{frame}{Modified Denavit-Hartenberg method}
	Example: 3 DoF (RRR) 
	
	\begin{minipage}{.4\columnwidth}
		$2^{\text nd}$ step: place $x$ axis at each such that $x_{j-1}\perp z_{j}$ and intersect with it\\
		%		\textbf{Remark:} the frame at the end-effector is set assuming a fixed joint 
	\end{minipage}
	\hfill
	\begin{minipage}{.5\columnwidth}
		\begin{figure}
			\centering
			\includegraphics[height=0.7\textheight]{ex-mDH-ZX-Axis}
		\end{figure}
	\end{minipage}
\end{frame}
\begin{frame}{Modified Denavit-Hartenberg method}
	Example: 3 DoF (RRR) 
	
	\begin{minipage}{.4\columnwidth}
		$3^{\text rd}$ step: determine the parameters
		\begin{table}
			\begin{tabular}{|c|c|c|c|c|c|c|}
				\hline
				$j$&$\sigma_j$&$a_j$&$\alpha_j$&$d_j$&$\theta_j$\\ \hline\hline
				$1$&$.$&$.$&$.$&$.$&$.$\\ \hline
				$2$&$.$&$.$&$.$&$.$&$.$\\ \hline
				$3$&$.$&$.$&$.$&$.$&$.$\\ \hline
				$ee$&$/$&$.$&$.$&$.$&$.$\\ \hline
			\end{tabular}
		\end{table} 
	\end{minipage}
	\hfill
	\begin{minipage}{.5\columnwidth}
		\begin{figure}
			\centering
			\includegraphics[height=0.7\textheight]{ex-mDH-ZX-Axis}
		\end{figure}
	\end{minipage}
\end{frame}
\begin{frame}{Modified Denavit-Hartenberg method}
	Example: 3 DoF (RRR) 
	
	\begin{minipage}{.4\columnwidth}
		$3^{\text rd}$ step: determine the parameters
		\begin{table}
			\begin{tabular}{|c|c|c|c|c|c|c|}
				\hline
				$j$&$\sigma_j$&$a_j$&$\alpha_j$&$d_j$&$\theta_j$\\ \hline\hline
				$1$&$0$&$0$&$0$&$l_1$&$q_1$\\ \hline
				$2$&$0$&$0$&$\pi/2$&$0$&$q_2$\\ \hline
				$3$&$0$&$l_2$&$0$&$0$&$q_3$\\ \hline
				$ee$&$/$&$l_3$&$0$&$0$&$0$\\ \hline
			\end{tabular}
		\end{table} 
	\end{minipage}
	\hfill
	\begin{minipage}{.5\columnwidth}
		\begin{figure}
			\centering
			\includegraphics[height=0.7\textheight]{ex-mDH-ZX-Axis}
		\end{figure}
	\end{minipage}
\end{frame}
\begin{frame}{Modified Denavit-Hartenberg method}
	Example: 3 DoF (RRR) 
	
	\begin{minipage}{.4\columnwidth}
		$3^{\text rd}$ step: determine the parameters
		\begin{table}
			\begin{tabular}{|c|c|c|c|c|c|c|}
				\hline
				$j$&$\sigma_j$&$a_j$&$\alpha_j$&$d_j$&$\theta_j$\\ \hline\hline
				$1$&$0$&$0$&$0$&$l_1$&$q_1$\\ \hline
				$2$&$0$&$0$&$\pi/2$&$0$&$q_2$\\ \hline
				$3$&$0$&$l_2$&$0$&$0$&$q_3$\\ \hline
				$ee$&$/$&$l_3$&$0$&$0$&$0$\\ \hline
			\end{tabular}
		\end{table} 
	\end{minipage}
	\hfill
	\begin{minipage}{.5\columnwidth}
		\begin{figure}
			\centering
			\includegraphics[height=0.7\textheight]{ex-mDH-ZX-Axis}
		\end{figure}
	\end{minipage}
\end{frame}

\begin{frame}{Modified Denavit-Hartenberg method}
	Example: 3 DoF (RRP) 
	
	\begin{minipage}{.4\columnwidth}
		$1^{\text st}$ step: place $z$ axis at each joint, $z_0$ axis is for the fixed frame\\
		\textbf{Remark:} the frame at the end-effector is set assuming a fixed joint 
	\end{minipage}
	\hfill
	\begin{minipage}{.5\columnwidth}
		\begin{figure}
			\centering
			\includegraphics[width=\columnwidth]{ex2-mDH-Z-Axis}
		\end{figure}
	\end{minipage}
\end{frame}
\begin{frame}{Modified Denavit-Hartenberg method}
	Example: 3 DoF (RRP) 
	
	\begin{minipage}{.4\columnwidth}
		$2^{\text nd}$ step: place $x$ axis at each such that $x_{j-1}\perp z_{j}$ and intersect with it\\
		%		\textbf{Remark:} the frame at the end-effector is set assuming a fixed joint 
	\end{minipage}
	\hfill
	\begin{minipage}{.5\columnwidth}
		\begin{figure}
			\centering
			\includegraphics[width=\columnwidth]{ex2-mDH-ZX-Axis}
		\end{figure}
	\end{minipage}
\end{frame}
\begin{frame}{Modified Denavit-Hartenberg method}
	Example: 3 DoF (RRP) 
	
	\begin{minipage}{.4\columnwidth}
		$3^{\text rd}$ step: determine the parameters
		\begin{table}
			\begin{tabular}{|c|c|c|c|c|c|c|}
				\hline
				$j$&$\sigma_j$&$a_j$&$\alpha_j$&$d_j$&$\theta_j$\\ \hline\hline
				$1$&$.$&$.$&$.$&$.$&$.$\\ \hline
				$2$&$.$&$.$&$.$&$.$&$.$\\ \hline
				$3$&$.$&$.$&$.$&$.$&$.$\\ \hline
				$ee$&$/$&$.$&$.$&$.$&$.$\\ \hline
			\end{tabular}
		\end{table} 
	\end{minipage}
	\hfill
	\begin{minipage}{.5\columnwidth}
		\begin{figure}
			\centering
			\includegraphics[width=\columnwidth]{ex2-mDH-ZX-Axis}
		\end{figure}
	\end{minipage}
\end{frame}
\begin{frame}{Modified Denavit-Hartenberg method}
	Example: 3 DoF (RRP) 
	
	\begin{minipage}{.4\columnwidth}
		$3^{\text rd}$ step: determine the parameters
		\begin{table}
			\begin{tabular}{|c|c|c|c|c|c|c|}
				\hline
				$j$&$\sigma_j$&$a_j$&$\alpha_j$&$d_j$&$\theta_j$\\ \hline\hline
				$1$&$0$&$0$&$0$&$0$&$q_1$\\ \hline
				$2$&$0$&$l_1$&$0$&$0$&$q_2$\\ \hline
				$3$&$1$&$l_2$&$0$&$0$&$0$\\ \hline
				$ee$&$/$&$0$&$\pi$&$q_3$&$0$\\ \hline
			\end{tabular}
		\end{table} 
	\end{minipage}
	\hfill
	\begin{minipage}{.5\columnwidth}
		\begin{figure}
			\centering
			\includegraphics[width=\columnwidth]{ex2-mDH-ZX-Axis}
		\end{figure}
	\end{minipage}
\end{frame}
\begin{frame}{Modified Denavit-Hartenberg method}
	Example: 3 DoF (RRP) 
	\begin{align*}
		T_1^0 &=\begin{bmatrix}
			\cos(q_1)&-\sin(q_1)&0&0\\
			\sin(q_1)&\cos(q_1)&0&0\\
			0&0&1&0\\
			0&0&0&1
		\end{bmatrix}, \ 
		T_2^1 =\begin{bmatrix}
			\cos(q_2)&-\sin(q_2)&0&l_1\\
			\sin(q_2)&\cos(q_2)&0&0\\
			0&0&1&0\\
			0&0&0&1
		\end{bmatrix},\\
		T_3^2 &=\begin{bmatrix}
			1&0&0&l_2\\
			0&1&0&0\\
			0&0&1&0\\
			0&0&0&1
		\end{bmatrix}, \ 
		T_{ee}^3 =\begin{bmatrix}
			1&0&0&0\\
			0&-1&0&0\\
			0&0&-1&-q_3\\
			0&0&0&1
		\end{bmatrix},
	\end{align*}
\end{frame}
\begin{frame}{Modified Denavit-Hartenberg method}
	Example: 3 DoF (RRP) 
	\begin{align*}
		T_{ee}^0 =\begin{bmatrix}
			\cos(q_1+q_2)&\sin(q_1+q_2)&0&l_1\cos(q_1)+l_2\cos(q_1+q_2)\\
			\sin(q_1+q_2)&-\cos(q_1+q_2)&0&l_1\sin(q_1)+l_2\sin(q_1+q_2)\\
			0&0&-1&-q_3\\
			0&0&0&1
		\end{bmatrix}
	\end{align*}
	Forward kinematics model of the end-effector: \\
	End-effector position: $\vectorP_{ee}= \begin{bmatrix}
		l_1\cos(q_1+q_2)&l_1\sin(q_1+q_2)&-q_3
	\end{bmatrix}^T$
	End-effector orientation: $\matriceR_{ee}= \begin{bmatrix}
	\cos(q_1+q_2)&\sin(q_1+q_2)&0\\
	\sin(q_1+q_2)&-\cos(q_1+q_2)&0\\
	0&0&-1
	\end{bmatrix}$
\end{frame}
\begin{frame}{Inverse kinematics}
	\begin{itemize}
		\item Forward kinematics model find the the end-effector pose $\vectorXMaj$ ($\vectorP_{ee},\matriceR_{ee}$) given the joints position $\vectorQ$: $\vectorX=f(\vectorQ)$ 		
		\item In control context, we are particularly interested in finding the joints position $\vectorQ$ given the end-effector pose $\vectorXMaj$: $\vectorQ=f^{-1}(\vectorQ)$
		\item This is the object of the \textbf{inverse kinematics model}
		\item There exist two approaches:
		\begin{enumerate}
			\item Analytical approach
			\item Numerical approach
		\end{enumerate}
%		\item In general, the forward kinematics model is nonlinear w.r.t $\vectorQ$ $\Rightarrow$ No systematic method to compute analytically $\vectorQ$
%		\item For some robots, the number of DoF is superieur than the dimension of the task $\Rightarrow$ redundant robot	
\end{itemize}	
\end{frame}
\begin{frame}{Inverse kinematics - Analytical approach}
Objective: find the analytically the solution of the inverse problem 

\begin{minipage}{.4\columnwidth}
Example:  a planar 2-DOF robot
\end{minipage}
\hfill
\begin{minipage}{.5\columnwidth}
	\begin{figure}
		\centering
		\includegraphics[width=\columnwidth]{2DOF-woAxis}
	\end{figure}
\end{minipage}	
\end{frame}
\begin{frame}{Inverse kinematics - Analytical approach}
	Objective: find the analytically the solution of the inverse problem 
	
	\begin{minipage}{.4\columnwidth}
		1st step: find the forward kinematic model
%		\begin{equation*}
%			\matriceT_{ee}^0=\begin{bmatrix}
%				\cos(q_1+q_2)&\sin(q_1+q_2)&0&l_1\cos(q_1)+l_2\cos(q_1+q_2)\\
%				\sin(q_1+q_2)&-\cos(q_1+q_2)&0&l_1\sin(q_1)+l_2\sin(q_1+q_2)\\
%				0&0&1&0\\
%				0&0&0&1
%			\end{bmatrix}
%		\end{equation*}
	\end{minipage}
	\hfill
	\begin{minipage}{.5\columnwidth}
		\begin{figure}
			\centering
			\includegraphics[width=\columnwidth]{2DOF-mDH-ZX-Axis}
		\end{figure}
	\end{minipage}	
\end{frame}
\begin{frame}{Inverse kinematics - Analytical approach}
	Objective: find the analytically the solution of the inverse problem \\
	1st step: find the forward kinematic model
	\begin{equation*}
	\matriceT_{ee}^0=\begin{bmatrix}
	\cos(q_1+q_2)&\sin(q_1+q_2)&0&l_1\cos(q_1)+l_2\cos(q_1+q_2)\\
	\sin(q_1+q_2)&-\cos(q_1+q_2)&0&l_1\sin(q_1)+l_2\sin(q_1+q_2)\\
	0&0&1&0\\
	0&0&0&1
	\end{bmatrix}
	\end{equation*}
	\begin{align*}
	p_{{ee}_x} &= l_1\cos(q_1)+l_2\cos(q_1+q_2)\\
	p_{{ee}_y} &= l_1\sin(q_1)+l_2\sin(q_1+q_2)
	\end{align*}
\end{frame}
\begin{frame}{Inverse kinematics - Analytical approach}
	Given $	p_{{ee}_x}$ and $p_{{ee}_y}$, find $q_1$ and $q_2$
	\begin{align*}
		p_{{ee}_x} &= l_1\cos(q_1)+l_2\cos(q_1+q_2)\\
		p_{{ee}_y} &= l_1\sin(q_1)+l_2\sin(q_1+q_2)
	\end{align*}%$\Leftrightarrow$
	\begin{align*}
		p_{{ee}_x}-l_1\cos(q_1) &= l_2\cos(q_1+q_2)\\
		p_{{ee}_y}-l_1\sin(q_1) &= l_2\sin(q_1+q_2)
	\end{align*}
	\begin{align*}
		\left(p_{{ee}_x}-l_1\cos(q_1)\right)^2 +\left(p_{{ee}_y}-l_1\sin(q_1)\right)^2 &= l_2^2\\
		\Leftrightarrow A\cos(q_1)+B\sin(q_1)+C&=0
	\end{align*}
\end{frame}
\begin{frame}{Inverse kinematics - Analytical approach}
	Two possible solutions:
	\begin{align*}
	q_{11} &= 2\arctan(\alpha_1),\ \alpha_1 = \frac{-B-\sqrt{A^2+B^2-C^2}}{C-A}\\
	q_{21}&= - q_{11} + \arctan2(\sin(q_1+q_2)/\cos(q_1+q_2))
	\end{align*}
	and 
	\begin{align*}
		q_{12} &= 2\arctan(\alpha_2),\ \alpha_1 = \frac{-B+\sqrt{A^2+B^2-C^2}}{C-A}\\
		q_{22}&= - q_{12} + \arctan2(\sin(q_1+q_2)/\cos(q_1+q_2))
	\end{align*}
\end{frame}
\begin{frame}{Inverse kinematics - Analytical approach}
	\begin{figure}
		\centering
		\includegraphics[width=0.47\columnwidth]{2DOF-Sol1}
		\includegraphics[width=0.47\columnwidth]{2DOF-Sol2}
		\caption{Two possible inverse kinematics solutions}
	\end{figure}
\end{frame}
\begin{frame}{Inverse kinematics - Analytical approach}
	\begin{itemize}
	\item In general, the forward kinematics model is nonlinear w.r.t $\vectorQ$ $\Rightarrow$ No systematic method to compute analytically $\vectorQ$
	\item If the robot is redundant (number of DoF is superieur than the dimension of the pose) $\Rightarrow$ inverse kinematics problem may have infinite solutions
	\item In such cases, we resort to numerical approaches	
	\end{itemize}
\end{frame}
\begin{frame}{Inverse kinematics - Numerical approach}

\end{frame}