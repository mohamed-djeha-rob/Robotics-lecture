%\begin{frame}{Multi-objective control notion}
%	\begin{figure}
%		\centering
%		\begin{tikzpicture}[auto, node distance=2cm,>=latex']
%			\node [block] (CRA) at (-1, 12) {Complex robotic application};
%			\node [sum] (comp) at (4,11) {}; 
%			\node [block] (adjust) at (6,11) {\color{red}Adjustment};
%			
%			
%		\end{tikzpicture}
%	\end{figure}
%\end{frame}
\begin{frame}
	\centering
	\Large Multi-Objective Control
\end{frame}
\subsection{Multi-Objective Control}
\begin{frame}{Redundancy}
	\begin{itemize}
		\item For some systems, the Jacobian matrix $  L_gL_f^{\rho-1}\vectorE(\vectorX)\inR^{p\times m}$ is wide: $p<m$
		\item Example: humanoid robot 
		\begin{itemize}
			\item Number of actuators $m=\sim30$
			\item Task: control the hand position $p=3$
		\end{itemize} 
		\item In this case, \begin{equation*}
			L_gL_f^{\rho-1}\vectorE(\vectorX)\vectorU =- L_f^\rho\vectorE(\vectorX) - \matriceK\vectorZ
		\end{equation*} may accept multiple $ \vectorU$ solutions\footnote[frame]{Namely, $\ker(L_gL_f^{\rho-1}\vectorE(\vectorX))\neq\emptyset$}, but all of theme acheive the task!
		\item This type of systems (robots) are called \emph{‘redundant’}: more DoF than required to perform a task
	\end{itemize}
\end{frame}
\begin{frame}{Redundancy}
	\begin{itemize}	
		\item \textbf{Idea:} From the set of all possible solutions, choose --\emph{if possible}-- the one that achieves \emph{simultaneously} other tasks 
		\item E.g., robotic arm
		\begin{itemize}
			\item Task 01: bring the hand to a desired position
			\item Task 02: keep the elbow up  
		\end{itemize}
		\item Why ‘if possible’? Depending on the system state, redundancy can be lost $\Rightarrow$ impossible to perform \emph{two tasks simultaneously}, only one task can be performed, e.g., for a robotic arm
		\begin{itemize}
			\item Task 01: bring the hand to a desired position and orientation
			\item Task 02: keep the elbow up
		\end{itemize}
		\item When two (or more) objectives cannot be met simultaneously (conflictual), a hierarchy needs to be defined
	\end{itemize}
\end{frame}
\begin{frame}{Tasks Hierarchy}
	\begin{itemize}
		\item What is the purpose of defining a hierarchy between tasks?
		\item Hierarchy defines how the redundancy should be resolved such that low-priority control objectives do not (or at least minimally) disturb the high-priority ones
		\item Two tasks hierarchical schemes exist:
%	\begin{itemize}
%		\item Strict
%		\item Soft/weighted
%	\end{itemize}
\end{itemize}
\begin{minipage}{.48\columnwidth}
	\begin{figure}
		\centering
		\includegraphics[height=0.38\textheight]{strict.png}
	%	\includegraphics[height=0.4\textheight]{weighted.png}content...
	\caption{Strict hierarchy}
	\end{figure}
\end{minipage}
\begin{minipage}{.48\columnwidth}
	\begin{figure}
		\centering
	%	\includegraphics[height=0.4\textheight]{strict.png}
		\includegraphics[height=0.4\textheight]{weighted.png}
		\caption{Soft/weighted hierarchy}
	\end{figure}
\end{minipage}
\end{frame}
\begin{frame}{Tasks Hierarchy}
	\textbf{Strict hierarchy:}
	\begin{itemize}
		 \item Strict order of priority is defined (lexicographic order)
		 \begin{equation*}
		 	{\rm Task~01}\succ{\rm Task~02}\succ\cdots\succ{\rm Task~N}
		 \end{equation*}
		 \item \textbf{Principle:} 
		 \begin{enumerate}
		 	\item Perform ${\rm Task~01}$;
		 	\item If redundancy is still available $\rightarrow$ perform ${\rm Task~02}$ such that ${\rm Task~01}$ is not disturbed; else keep solution in (1)
		 	\item If redundancy is still available $\rightarrow$ perform ${\rm Task~03}$ such that ${\rm Task~01}$ and ${\rm Task~02}$ are not disturbed; else keep solution in (2)...
		 \end{enumerate}
		 \item \textbf{Methodology:} Solve cascade of QPs
		 \begin{itemize}
		 	\item \textbf{Pros:} Helps solving infeasibility
		 	\item \textbf{Cons:} Time consuming, not all tasks are performed, order subjectivity
		 \end{itemize}	  
	\end{itemize}
\end{frame}
\begin{frame}{Tasks Hierarchy}
	\textbf{Soft hierarchy:}
	\begin{itemize}
		\item Tasks are sorted by scalar positive weights 
		\begin{align}\label{eq:cost-func weighted QP}
			w_1\norm{\matriceA_1\vectorU+\vectorB_1}^2 &+ w_2\norm{\matriceA_2\vectorU+\vectorB_2}^2 + \cdots +
			w_N\norm{\matriceA_N\vectorU+\vectorB_N}^2,\\
			&w_i\geq0,\ i\in\left\{1,\cdots,N\right\}
		\end{align}
		\item \textbf{Principle:} Solve redundancy such that each task is performed \emph{at best} according to its weight (tasks competition)
		\item \textbf{Methodology:} Solve only one QP which cost-function is~\eqref{eq:cost-func weighted QP}
		\begin{itemize}
			\item \textbf{Pros:} Solved efficiently, all tasks are performed proportionally to their weights
			\item \textbf{Cons:} Precision not guaranteed, potential infeasibility due to constraints conflict 
		\end{itemize}
	\end{itemize}
\end{frame}
\begin{frame}{Tasks Hierarchy}
	\begin{tcolorbox}[colback=red!5!white, colframe=red!50!black, title=Remark]
		In both strict and soft hierarchical schemes: 
		\begin{itemize}
			\item the constraints have  \textbf{always} a higher priority than the tasks within the QP structure
			\item The constraints have the same level of priority
		\end{itemize}
	\end{tcolorbox}
	
\end{frame}
%\begin{frame}{Multi-objective control - Example}
%	Example: Moving the base of a quadruped robot
%	\begin{minipage}{0.51\columnwidth}
%		\begin{figure}
%			\includegraphics[width=\columnwidth]{multi-task example.pdf}
%		\end{figure}
%	\end{minipage}
%	\hfill
%	\begin{minipage}{0.48\columnwidth}
%		\begin{itemize}
%			\item \textbf{Task:} move the base in the manipulability space
%			\item \textbf{Constraints:} 
%			\begin{itemize}
%				\item Keep end-effector fixed
%				\item Keep non-slipping contacts
%				\item Avoid self-collision
%				\item Respect joint position/velocity/torque constraints
%				\item Keep CoM projection inside the static equilibrium region
%			\end{itemize}
%		\end{itemize}
%		
%	\end{minipage}
%\end{frame}